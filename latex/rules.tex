\documentclass[11pt,a4paper]{article}
%twocolumn
\usepackage[utf8]{inputenc}
\usepackage[T1]{fontenc}
\usepackage{lmodern}
\usepackage{tikz}
\usepackage[margin=1.2in]{geometry}
\usepackage{amsmath}
\usepackage{mathtools}
\usepackage{hyperref}
\usepackage{amssymb}
\usepackage{latexsym}
\usepackage{syntax}
\usepackage{lscape}
\usepackage{stmaryrd}
\usepackage{microtype}
\usepackage{graphicx}
\usepackage{mathpartir}
%\usepackage[pdftex]{graphicx}

%\usetikzlibrary{positioning,calc}

\DeclareMathSymbol{\mlq}{\mathord}{operators}{``}
\DeclareMathSymbol{\mrq}{\mathord}{operators}{`'}
\DeclareMathOperator*{\argmin}{\arg\!\min}
\DeclareMathOperator*{\argmax}{\arg\!\max}


\def\extraVskip{3pt}
\newenvironment{scprooftree}[1]%
  {\gdef\scalefactor{#1}\begin{center}\proofSkipAmount \leavevmode}%
  {\scalebox{\scalefactor}{\DisplayProof}\proofSkipAmount \end{center} }

\makeatletter
\providecommand{\bigsqcap}{%
  \mathop{%
    \mathpalette\@updown\bigsqcup
  }%
}
\newcommand*{\@updown}[2]{%
  \rotatebox[origin=c]{180}{$\m@th#1#2$}%
}
\makeatother




\begin{document}
\pagenumbering{arabic}

\newcommand{\Heap}{H}

\newcommand{\sfrme}{\ensuremath{\vdash_\texttt{sfrm}}}
\newcommand{\sfrmphi}{\ensuremath{\vdash_\texttt{sfrm}}}
\newcommand{\true}{\ensuremath{\texttt{true}}}
\newcommand{\vnull}{\ensuremath{\texttt{null}}}
\newcommand{\Tint}{\ensuremath{\texttt{int}}}
\newcommand{\xresult}{\ensuremath{\texttt{result}}}
\newcommand{\xthis}{\ensuremath{\texttt{this}}}
\newcommand{\new}{\ensuremath{\texttt{new}~}}
\newcommand{\assert}{\ensuremath{\texttt{assert}~}}
\newcommand{\release}{\ensuremath{\texttt{release}~}}
\newcommand{\return}{\ensuremath{\texttt{return}~}}
\newcommand{\acc}{\ensuremath{\texttt{acc}}}
\newcommand{\fields}{\ensuremath{\texttt{fields}}}
\newcommand{\mpre}{\ensuremath{\texttt{mpre}}}
\newcommand{\mpost}{\ensuremath{\texttt{mpost}}}
\newcommand{\mbody}{\ensuremath{\texttt{mbody}}}
\newcommand{\mparam}{\ensuremath{\texttt{mparam}}}
\newcommand{\mrettype}{\ensuremath{\texttt{mrettype}}}
\newcommand{\mmethod}{\ensuremath{\texttt{mmethod}}}

%\newcommand{\staticFP}[1]{\ensuremath{\texttt{static-footprint}(#1)}}
%\newcommand{\dynamicFP}[3]{\ensuremath{\texttt{footprint}_{#1,#2}(#3)}}

\newcommand\floor[1]{\lfloor#1\rfloor}
\newcommand\ceil[1]{\lceil#1\rceil}
\newcommand{\staticFP}[1]{\ensuremath{\floor{#1}}}
\newcommand{\dynamicFP}[3]{\ensuremath{\floor{#3}_{#1,#2}}}

\newcommand{\rlabel}[1]{\RightLabel{\quad #1}}
\newcommand{\dom}{\ensuremath{\texttt{dom}}}

\newcommand{\class}{\ensuremath{\texttt{class}~}}
\newcommand{\requires}{\ensuremath{\texttt{requires}~}}
\newcommand{\ensures}{\ensuremath{\texttt{ensures}~}}

\section{Syntax}

\begin{align*}
\\ &program    	&&::= \overline{cls}~\overline{s}
\\ &cls    		&&::= \class C~\{\overline{field}~\overline{method}\}
\\ &field    	&&::= T~f;
\\ &method		&&::= T~m(T~x)~contract~\{\overline{s}\}
\\ &contract	&&::= \requires \phi;~\ensures \phi;
\\ &T			&&::= \Tint ~|~ C
\\ &s			&&::= x.f := y;
				  ~|~ x := e; 
				  ~|~ x := \new C; 
				  ~|~ x := y.m(z);
\\ & &&
				  ~|~ \return x; 
				  ~|~ \assert \phi; 
				  ~|~ \release \phi;
				  ~|~ T~x;
\\ &\phi		&&::= \true
				  ~|~ e = e
				  ~|~ e \neq e
				  ~|~ \acc(x.f)
				  ~|~ x : T
				  ~|~ \phi * \phi
\\ &e			&&::= v
				  ~|~ x
				  ~|~ e.f
\\ &x			&&::= \xthis ~|~ \xresult ~|~ \langle other \rangle
\\ &v			&&::= o ~|~ n ~|~ \vnull
\\ &n			&&\in~~ \mathbb{Z}
\\				  
\\ &H			&&\in~~ (o \rightharpoonup (C,\overline{(f \rightharpoonup v)}))
\\ &\rho		&&\in~~ (x \rightharpoonup v)
\\ &A_s			&&::= \overline{(x, f)}
\\ &A_d			&&::= \overline{(o, f)}
\\ &S			&&::= (\rho, A_d, \overline{s}) \cdot S ~|~ nil
\end{align*}

\section{Static semantics}
\subsection{Expressions ($A_s \sfrme e$)}
\begin{mathpar}
\inferrule* [right=WFVar]
{~}
{A \sfrme x}
\end{mathpar}\hfill

\begin{mathpar}
\inferrule* [right=WFValue]
{~}
{A \sfrme v}
\end{mathpar}\hfill

\begin{mathpar}
\inferrule* [right=WFField]
{( x , f ) \in A}
{A \sfrme x.f}
\end{mathpar}\hfill



\subsection{Formulas ($A_s \sfrme \phi$)}
% Inductive Semantics.sfrmphi'
\begin{mathpar}
\inferrule* [Right=WFTrue]
{~}
{{A} \sfrmphi {\true}}
\end{mathpar}

\begin{mathpar}
\inferrule* [Right=WFEqual]
{{A} \sfrme {e_1}\\{A} \sfrme {e_2}}
{{A} \sfrmphi {({e_1} = {e_2})}}
\end{mathpar}

\begin{mathpar}
\inferrule* [Right=WFNEqual]
{{A} \sfrme {e_1}\\{A} \sfrme {e_2}}
{{A} \sfrmphi {({e_1} \neq {e_2})}}
\end{mathpar}

\begin{mathpar}
\inferrule* [Right=WFAcc]
{{A} \sfrme {e}}
{{A} \sfrmphi {\acc({e}.{f})}}
\end{mathpar}

\begin{mathpar}
\inferrule* [Right=WFType]
{~}
{{A} \sfrmphi {{x} : {T}}}
\end{mathpar}



\begin{mathpar}
\inferrule* [Right=WFSepOp]
{
A_s \sfrmphi \phi_1 \\ 
A_s \cup \staticFP {\phi_1} \sfrmphi \phi_2
}
{A_s \sfrmphi \phi_1 * \phi_2}
\end{mathpar}

\subsubsection{Implication ($\phi_1 \dot{\implies} \phi_2$)}
Conservative approx. of $\phi_1 \implies \phi_2$.

\subsection{Footprint ($\staticFP {\phi} = A_s$)}
\begin{align*}
 &\staticFP {\true}    		&&= \emptyset
\\ &\staticFP {e_1 = e_2}     	&&= \emptyset
\\ &\staticFP {e_1 \neq e_2}  	&&= \emptyset
\\ &\staticFP {\acc(x.f)} 		&&= \{(x,f)\}
\\ &\staticFP {\phi_1 * \phi_2} 	&&= \staticFP {\phi_1} \cup \staticFP {\phi_2}
\end{align*}

\newcommand{\sType}[3]{#1 \vdash #2 : #3}
\subsection{Type ($\sType {\phi} {e} {T}$)}
\begin{mathpar}
\inferrule* [Right=STValue]
{~}
{\sType {\phi} {v_T} {T}}
\end{mathpar}

\begin{mathpar}
\inferrule* [Right=STVar]
{\phi \implies (x : T)}
{\sType {\phi} {x} {T}}
\end{mathpar}

\begin{mathpar}
\inferrule* [Right=STField]
{ \sType {\phi} {e} {C}
\\ \vdash C.f : T}
{\sType {\phi} {e.f} {T}}
\end{mathpar}

\newcommand{\hoare}[3]{\vdash\{#1\}#2\{#3\}}
\subsection{Hoare ($\hoare {\phi} {\overline s} {\phi}$)}
\begin{mathpar}
\inferrule* [Right=HSec]
{
\hoare {\phi_p} {s_1} {\phi_{q1}} \\ 
\phi_{q1} \implies \phi_{q2} \\
\hoare {\phi_{q2}} {s_2} {\phi_r}
}
{\hoare {\phi_p} {s_1;s_2} {\phi_r}}
\end{mathpar}

\begin{mathpar}
\inferrule* [Right=HNewObj]
{\phi\vdash{x}:{C}\\ \fields(C) = \overline{f} }
{\hoare {\phi} {x{\::=\:\new\:}C} {( \acc(x,\overline{f_i}) * ( (x{\:\neq\:}\vnull) * \phi ) )}}
\end{mathpar}

\begin{mathpar}
\inferrule* [Right=HFieldAssign]
{\phi\vdash{x}:{C}\\\vdash{C.f}:{T}\\\phi\vdash{y}:{T}\\\phi{\:\implies\:}\acc(x,f)\\\phi{\:\implies\:}(x{\:\neq\:}\vnull)}
{\hoare {\phi} {x.f{\::=\:}y} {\phi{\:*\:}(x.f{\:=\:}y)}}
\end{mathpar}

\begin{mathpar}
\inferrule* [Right=HVarAssign]
{\phi_1\vdash{x}:{T}\\\phi_1\vdash{e}:{T}\\ \phi_1 = \phi_2[e/x] \\\emptyset \sfrmphi \phi_1\\\staticFP{\phi_1} \sfrme e}
{\hoare {\phi_1} {x{\::=\:}e} {\phi_2}}
\end{mathpar}

\begin{mathpar}
\inferrule* [Right=HReturn]
{\phi\vdash{x}:{T}\\\phi\vdash{\xresult}:{T}}
{\hoare {\phi} {{\return}x} {\phi{\:*\:}(\xresult{\:=\:}x)}}
\end{mathpar}

\begin{mathpar}
\inferrule* [Right=HApp]
{\phi\vdash{y}:{C}\\\phi\vdash{x}:{T_r}\\\phi\vdash{z'}:{T_p}\\\phi{\:\implies\:}(y{\:\neq\:}\vnull)\\\phi{\:\implies\:}( \phi_p * \phi_r )\\ \mmethod(C,m) = T_r~m(T_p~z)~\requires{\phi_{pre}};~\ensures{\phi_{post}};~\{\_\} \\ \phi_p = \phi_{pre}[y,z'/\xthis,z] \\ \phi_q = \phi_{post}[y,z',x/\xthis,z,\xresult] }
{\hoare {\phi} {x{\::=\:}y.m(z')} {( \phi_q * \phi_r )}}
\end{mathpar}

\begin{mathpar}
\inferrule* [Right=HAssert]
{\phi_1{\:\implies\:}\phi_2}
{\hoare {\phi_1} {{\assert}\phi_2} {\phi_1}}
\end{mathpar}

\begin{mathpar}
\inferrule* [Right=HRelease]
{\phi_1{\:\implies\:}( \phi_2 * \phi_r )\\\emptyset \sfrmphi \phi_r}
{\hoare {\phi_1} {{\release}\phi_2} {\phi_r}}
\end{mathpar}

\begin{mathpar}
\inferrule* [Right=HDeclare]
{x\not\in \dom(\phi_1)\\ \phi_2 = \phi_1{\:*\:}x{\::\:}T{\:*\:}(x{\:=\:}\texttt{defaultValue}(T)) }
{\hoare {\phi_1} {T{\:}x} {\phi_2}}
\end{mathpar}



\section{Dynamic semantics}
\newcommand{\evalex}[4]{#1,#2 \vdash #3 \Downarrow #4}
\newcommand{\evale}[2]{H,\rho \vdash #1 \Downarrow #2}
\subsection{Expressions ($\evale {e} {v}$)}

\begin{mathpar}
\inferrule* [Right=EEVar]
{~}
{\evale {x} {\rho(x)}}
\end{mathpar}

\begin{mathpar}
\inferrule* [Right=EEValue]
{~}
{\evale {v} {v}}
\end{mathpar}

\begin{mathpar}
\inferrule* [Right=EEAcc]
{\evale {x} {o}}
{\evale {x.f} {H(o)(f)}}
\end{mathpar}

\newcommand{\evalphix}[4]{#1,#2,#3 \vDash #4}
\newcommand{\evalphi}{\evalphix H \rho A}
\subsection{Formulas ($\evalphi \phi$)}
\begin{mathpar}
\inferrule* [right=EATrue]
{~}
{\evalphix {h} {r} {a} {\true}}
\end{mathpar}\hfill

\begin{mathpar}
\inferrule* [right=EAEqual]
{ \evale {h} {r} {e1} = v1  \\  \evale {h} {r} {e2} = v2  \\  v1 = v2 }
{\evalphix {h} {r} {a} {e1{\:=\:}e2}}
\end{mathpar}\hfill

\begin{mathpar}
\inferrule* [right=EANEqual]
{ \evale {h} {r} {e1} = v1  \\  \evale {h} {r} {e2} = v2  \\  v1 <> v2 }
{\evalphix {h} {r} {a} {e1{\:\neq\:}e2}}
\end{mathpar}\hfill

\begin{mathpar}
\inferrule* [right=EAAcc]
{ r  x = (vo  o')  \\ ( o' , f' ) \in a}
{\evalphix {h} {r} {a} {\acc(x,f')}}
\end{mathpar}\hfill



\begin{mathpar}
\inferrule* [Right=EASepOp]
{
A_1 = A \backslash A_2 \\
\evalphix H \rho {A_1} {\phi_1} \\
\evalphix H \rho {A_2} {\phi_2}
}
{\evalphi {\phi_1 * \phi_2}}
\end{mathpar}


\subsubsection{Implication ($\phi_1 \implies \phi_2$)}
\begin{equation*}
\phi_1 \implies \phi_2
\quad\quad \iff \quad\quad
\forall H, \rho, A: \evalphi \phi_1 \implies \evalphi \phi_2
\end{equation*}
Drawn from def. of entailment in ``A Formal Semantics for Isorecursive
and Equirecursive State Abstractions''.

\subsection{Footprint ($\dynamicFP {H} {\rho} {\phi} = A_d$)}
\begin{align*}
 &\dynamicFP {H} {\rho} {\true}    		&&= \emptyset
\\ &\dynamicFP {H} {\rho} {e_1 = e_2}     	&&= \emptyset
\\ &\dynamicFP {H} {\rho} {e_1 \neq e_2}  	&&= \emptyset
\\ &\dynamicFP {H} {\rho} {\acc(e.f)} 		&&= \{(o,f)\} \text{ where } \evale e o
\\ &\dynamicFP {H} {\rho} {\phi_1 * \phi_2} &&= \dynamicFP {H} {\rho} {\phi_1} \cup \dynamicFP {H} {\rho} {\phi_2}
\end{align*}

\newcommand{\dType}[4]{#1, #2 \vdash #3 : #4}
\subsection{Type ($\dType {H} {\rho} {e} {T}$)}
\begin{mathpar}
\inferrule* [Right=DTEval]
{ \evale {e} {v_T} }
{\dType {H} {\rho} {e} {T}}
\end{mathpar}

\newcommand{\sstepGeneric}[5]{({#1}, {#2}) \rightarrow^{#3} ({#4}, {#5})}
\newcommand{\sstep}[4]{\sstepGeneric {#1} {#2} {} {#3} {#4}}
\newcommand{\sstepM}[4]{\sstepGeneric {#1} {#2} * {#3} {#4}} 
\newcommand{\sstepWS}[4]{\sstepGeneric {#1} {{#2} \cdot S} {} {#3} {{#4} \cdot S}}
\newcommand{\sstepWSX}[8]{\sstepGeneric {#1} {({#2},{#3},{#4}) \cdot S} {} {#5} {({#6},{#7},{#8}) \cdot S}}

\newcommand{\Tfs}{\overline{T}~\overline{f}}
\subsection{Small-step ($\sstep H S H S$)}
\begin{mathpar}
\inferrule* [right=ESFieldAssign]
{\evalex {\Heap} {\rho} {x} {o}\\\evalex {\Heap} {\rho} {y} {v_y}\\\textcolor{gray}{( o , f ) \in A}\\ \Heap' = \Heap[o{\:\mapsto\:}[f{\:\mapsto\:}v_y]] }
{{( \Heap , ( \rho ,\textcolor{gray}{ A }, x.f{\::=\:}y; \overline{s} ) \cdot S )} \rightarrow {( \Heap' , ( \rho ,\textcolor{gray}{ A }, \overline{s} ) \cdot S )}}
\end{mathpar}\hfill

\begin{mathpar}
\inferrule* [right=ESVarAssign]
{\evalex {\Heap} {\rho} {e} {v}\\ \rho' = \rho[x{\:\mapsto\:}v] }
{{( \Heap , ( \rho ,\textcolor{gray}{ A }, x{\::=\:}e; \overline{s} ) \cdot S )} \rightarrow {( \Heap , ( \rho' ,\textcolor{gray}{ A }, \overline{s} ) \cdot S )}}
\end{mathpar}\hfill

\begin{mathpar}
\inferrule* [right=ESNewObj]
{ \Heap(o) = \bot \\ \fields(C) = f \\ \rho' = \rho[x{\:\mapsto\:}o] \\\textcolor{gray}{ A' = A * \overline{ ( o , f_i ) } }\\ \Heap' = \Heap[o{\:\mapsto\:}[(\overline{ ( f_i , \vnull ) })]] }
{{( \Heap , ( \rho ,\textcolor{gray}{ A }, x{\::=\:\new\:}C; \overline{s} ) \cdot S )} \rightarrow {( \Heap' , ( \rho' ,\textcolor{gray}{ A' }, \overline{s} ) \cdot S )}}
\end{mathpar}\hfill

\begin{mathpar}
\inferrule* [right=ESReturn]
{\evalex {\Heap} {\rho} {x} {v_x}\\ \rho' = \rho[\xresult{\:\mapsto\:}v_x] }
{{( \Heap , ( \rho ,\textcolor{gray}{ a }, {\return}x; \overline{s} ) \cdot S )} \rightarrow {( \Heap , ( \rho' ,\textcolor{gray}{ a }, \overline{s} ) \cdot S )}}
\end{mathpar}\hfill

\begin{mathpar}
\inferrule* [right=ESApp]
{\evalex {\Heap} {\rho} {y} {o}\\\evalex {\Heap} {\rho} {z} {v}\\ \Heap(o) = ( C , fvf ) \\ \mbody(C,m) = \overline{r} \\ \mparam(C,m) = ( T , w ) \\\textcolor{gray}{ \mpre(C,m) = \phi }\\ \rho' = [\xthis{\:\mapsto\:}o,w{\:\mapsto\:}v] \\\textcolor{gray}{\evalphix {\Heap} {\rho'} {A} {\phi}}\\\textcolor{gray}{ A' = \texttt{footprint}_{\Heap,\rho'}(\phi) }}
{{( \Heap , ( \rho ,\textcolor{gray}{ A }, x{\::=\:}y.m(z); \overline{s} ) \cdot S )} \rightarrow {( \Heap , ( \rho' ,\textcolor{gray}{ A' }, \overline{r} ) * ( \rho ,\textcolor{gray}{ A{\:\backslash\:}A' }, x{\::=\:}y.m(z); \overline{s} ) \cdot S )}}
\end{mathpar}\hfill

\begin{mathpar}
\inferrule* [right=ESAppFinish]
{\textcolor{gray}{ \mpost(C,m) = \phi }\\\textcolor{gray}{\evalphix {\Heap} {\rho'} {A'} {\phi}}\\\textcolor{gray}{ A'' = \texttt{footprint}_{\Heap,\rho'}(\phi) }\\\evalex {\Heap} {\rho'} {\xresult} {v_r}}
{{( \Heap , ( \rho' ,\textcolor{gray}{ A' }, \emptyset ) * ( \rho ,\textcolor{gray}{ A }, x{\::=\:}y.m(z); \overline{s} ) \cdot S )} \rightarrow {( \Heap , ( \rho[x{\:\mapsto\:}v_r] ,\textcolor{gray}{ A * A'' }, \overline{s} ) \cdot S )}}
\end{mathpar}\hfill

\begin{mathpar}
\inferrule* [right=ESAssert]
{\textcolor{gray}{\evalphix {\Heap} {\rho} {A} {\phi}}}
{{( \Heap , ( \rho ,\textcolor{gray}{ A }, {\assert}\phi; \overline{s} ) \cdot S )} \rightarrow {( \Heap , ( \rho ,\textcolor{gray}{ A }, \overline{s} ) \cdot S )}}
\end{mathpar}\hfill

\begin{mathpar}
\inferrule* [right=ESRelease]
{\textcolor{gray}{\evalphix {\Heap} {\rho} {A} {\phi}}\\\textcolor{gray}{ A' = A{\:\backslash\:}\texttt{footprint}_{\Heap,\rho}(\phi) }}
{{( \Heap , ( \rho ,\textcolor{gray}{ A }, {\release}\phi; \overline{s} ) \cdot S )} \rightarrow {( \Heap , ( \rho ,\textcolor{gray}{ A' }, \overline{s} ) \cdot S )}}
\end{mathpar}\hfill


%% Inductive Semantics.dynSem
\begin{mathpar}
\inferrule* [Right=ESFieldAssign]
{\evalex {H} {\rho} {{x}} {{o}}\\\evalex {H} {\rho} {{y}} {v_y}\\{({o}, {f})} \in {A}\\{H'} = {{H}[{o} \mapsto [{f} \mapsto {v_y}]]}}
{{({H}, {{({{\rho}, {A}}, {{{x}.{f} := {y}} \cdot {\overline{s}}})} \cdot {S}})} \rightarrow {({H'}, {{({{\rho}, {A}}, {\overline{s}})} \cdot {S}})}}
\end{mathpar}

\begin{mathpar}
\inferrule* [Right=ESVarAssign]
{\evalex {H} {\rho} {e} {v}\\{\rho'} = {{\rho}[{x} \mapsto {v}]}}
{{({H}, {{({{\rho}, {A}}, {{{x} := {e}} \cdot {\overline{s}}})} \cdot {S}})} \rightarrow {({H}, {{({{\rho'}, {A}}, {\overline{s}})} \cdot {S}})}}
\end{mathpar}

\begin{mathpar}
\inferrule* [Right=ESNewObj]
{{o} \not\in \dom({H})\\{\fields({C})} = {{Tfs}}\\{\rho'} = {{\rho}[{x} \mapsto {{o}}]}\\{A'} = {@app(prod(o, f), A, @map(prod(T, f), prod(o, f), fun(cf', :, prod, T, f, =>, @pair, o, f, o, @snd(T, f, cf')), Tfs))}\\{H'} = {{H}[{o} \mapsto [\overline{f \mapsto \texttt{defaultValue}(T)}]]}}
{{({H}, {{({{\rho}, {A}}, {{{x} := \new {C}} \cdot {\overline{s}}})} \cdot {S}})} \rightarrow {({H'}, {{({{\rho'}, {A'}}, {\overline{s}})} \cdot {S}})}}
\end{mathpar}

\begin{mathpar}
\inferrule* [Right=ESReturn]
{\evalex {H} {\rho} {{x}} {v_x}\\{\rho'} = {{\rho}[{\xresult} \mapsto {v_x}]}}
{{({H}, {{({{\rho}, {A}}, {{\return {x}} \cdot {\overline{s}}})} \cdot {S}})} \rightarrow {({H}, {{({{\rho'}, {A}}, {\overline{s}})} \cdot {S}})}}
\end{mathpar}

\begin{mathpar}
\inferrule* [Right=ESApp]
{\evalex {H} {\rho} {{y}} {{o}}\\\evalex {H} {\rho} {{z}} {v}\\{H(o)} = {{({C}, {\_})}}\\{\mmethod({C}, {m})} = {{{T_r}~{m}({T}~{w})~{\requires {phi};~\ensures {\_};}~\{ {\overline{r}} \}}}\\{\rho'} = {[{\xresult} \mapsto {\texttt{defaultValue}({T_r})}, {\xthis} \mapsto {{o}}, {{w}} \mapsto {v}]}\\\evalphix {H} {\rho'} {A} {phi}\\{A'} = {\dynamicFP {H} {\rho'} {phi}}}
{{({H}, {{({{\rho}, {A}}, {{{x} := {y}.{m}({z})} \cdot {\overline{s}}})} \cdot {S}})} \rightarrow {({H}, {{({{\rho'}, {A'}}, {\overline{r}})} \cdot {{({({\rho}, {{A} \backslash {A'}})}, {{{x} := {y}.{m}({z})} \cdot {\overline{s}}})} \cdot {S}}})}}
\end{mathpar}

\begin{mathpar}
\inferrule* [Right=ESAppFinish]
{\evalex {H} {\rho} {{y}} {{o}}\\{H(o)} = {{({C}, {\_})}}\\{\mpost({C}, {m})} = {{\phi}}\\\evalphix {H} {\rho'} {A'} {\phi}\\{A''} = {\dynamicFP {H} {\rho'} {\phi}}\\\evalex {H} {\rho'} {{\xresult}} {v_r}}
{{({H}, {{({{\rho'}, {A'}}, {\emptyset})} \cdot {{({{\rho}, {A}}, {{{x} := {y}.{m}({z})} \cdot {\overline{s}}})} \cdot {S}}})} \rightarrow {({H}, {{({({{\rho}[{x} \mapsto {v_r}]}, {@app(prod(o, f), A, A'')})}, {\overline{s}})} \cdot {S}})}}
\end{mathpar}

\begin{mathpar}
\inferrule* [Right=ESAssert]
{\evalphix {H} {\rho} {A} {\phi}}
{{({H}, {{({{\rho}, {A}}, {{\assert {\phi}} \cdot {\overline{s}}})} \cdot {S}})} \rightarrow {({H}, {{({{\rho}, {A}}, {\overline{s}})} \cdot {S}})}}
\end{mathpar}

\begin{mathpar}
\inferrule* [Right=ESRelease]
{\evalphix {H} {\rho} {A} {\phi}\\{A'} = {{A} \backslash {\dynamicFP {H} {\rho} {\phi}}}}
{{({H}, {{({{\rho}, {A}}, {{\release {\phi}} \cdot {\overline{s}}})} \cdot {S}})} \rightarrow {({H}, {{({({\rho}, {A'})}, {\overline{s}})} \cdot {S}})}}
\end{mathpar}

\begin{mathpar}
\inferrule* [Right=ESDeclare]
{{\rho'} = {{\rho}[{x} \mapsto {\texttt{defaultValue}({T})}]}}
{{({H}, {{({{\rho}, {A}}, {{{T}~{x}} \cdot {\overline{s}}})} \cdot {S}})} \rightarrow {({H}, {{({{\rho'}, {A}}, {\overline{s}})} \cdot {S}})}}
\end{mathpar}



\section{Gradualization}
\subsection{Syntax}
\begin{align*}
&\widetilde{\phi} \quad ::= \quad \phi ~|~ \phi\:*\:?
\end{align*}
Note: allowing $?$ at different position is hardly useful (no difference for dyn. semantics, useless difference in static semantics).

\subsection{Concretization A}
\begin{align*}
&\gamma(\phi) ~&&= \{~ \phi' ~|~ \phi' \iff \phi ~\} \\
&\gamma(\phi\:*\:?) ~&&= \{~ \phi' ~|~ \exists \phi_x : \phi\:*\:\phi_x \iff \phi' ~\} \\
&  ~&&= \{~ \phi' ~|~ \phi' \implies \phi ~\} \\
\end{align*}

\subsection{Abstraction (to show: set of $\phi$s is poset - can we even say that on infinite sets?)}
\begin{align*}
&\alpha(\overline{\phi}) = (\sqcap \overline{\phi})\:*\:? \\
\end{align*}

\subsection{Concretization B}
\begin{align*}
&\gamma(\phi) ~&&= \{~ \phi ~\} \\
&\gamma(\phi\:*\:?) ~&&= \{~ \phi\:*\:\phi_x ~|~ \phi_x ~\} \\
\end{align*}

\subsection{Abstraction}
\begin{align*}
&\alpha(\{~ \phi ~\}) = \phi \\
&\alpha(\{~ \overline{\phi} ~\}) = \texttt{lcp}(\overline{\phi}) \\
\end{align*}

\subsection{Theorems}
\subsubsection{Soundness}
\begin{align*}
&\forall \overline{\phi} : \overline{\phi} \subseteq \gamma(\alpha(\overline{\phi}))
\end{align*}
\subsubsection{Optimality}
\begin{align*}
&\forall \overline{\phi}, \widetilde{\phi} : \overline{\phi} \subseteq \gamma(\widetilde{\phi}) \implies \gamma(\widetilde{\phi}) \subseteq \gamma(\alpha(\overline{\phi}))
\end{align*}

\subsection{Concretization W (as in wrong)}
\begin{align*}
&\gamma(\phi) ~&&= \{~ (H,\rho,A) ~|~ \evalphi \phi ~\} \\
&\gamma(\phi\:*\:?) ~&&= \{~ (H,\rho,A) ~|~ \evalphi \phi ~\} \\
\end{align*}


\section{Theorems}
\subsection{Invariant $invariant(H, \rho, A_d, \phi)$}
\subsubsection{Heap consistent}
\begin{align*}
\forall x, o, C : 
	\rho(x) = o_C &\implies\\
      \exists f_C, m :&~ 
        \texttt{fields}(C) = f_C \\
      \wedge&~ H(o_C) = (C, m) \\
      \wedge&~(\forall (T, f) \in f_C : \dType {H} {\rho} {m(f)} {T})
\end{align*}

\subsubsection{Phi holds}
\begin{align*}
    \evalphix {H} {\rho} {A_d} {\phi}
\end{align*}

\subsubsection{Types preserved}
\begin{align*}
    \forall e, T : \sType {\phi} {e} {T}& \\
    \implies \dType {H} {\rho} {e} {T}&
\end{align*}

\subsection{Soundness}
\subsubsection{Progress}
\begin{align*}
\forall~ ... &:~~ \hoare {\phi_1} {s'} {\phi_2} 
\\ &\implies invariant(H_1, \rho_1, A_1, \phi_1)
\\ &\implies \exists H_2, \rho_2, A_2 : (H_1, (\rho_1, A_1, s' ; \overline{s}) \cdot S)
							\rightarrow^* (H_2, (\rho_2, A_2, \overline{s}) \cdot S)
\end{align*}

\subsubsection{Preservation}
\begin{align*}
\forall~ ... &:~~ \hoare {\phi_1} {s'} {\phi_2} 
\\ &\implies invariant(H_1, \rho_1, A_1, \phi_1)
\\ &\implies (H_1, (\rho_1, A_1, s' ; \overline{s}) \cdot S)
  \rightarrow^* (H_2, (\rho_2, A_2, \overline{s}) \cdot S)
\\ &\implies invariant(H_2, \rho_2, A_2, \phi_2)
\end{align*}

\end{document}