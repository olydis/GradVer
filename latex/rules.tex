\documentclass[11pt,a4paper]{article}
%twocolumn
\usepackage[utf8]{inputenc}
\usepackage[T1]{fontenc}
\usepackage{lmodern}
\usepackage{tikz}
\usepackage[margin=0.5in]{geometry}
\usepackage{amsmath}
\usepackage{mathtools}
\usepackage{hyperref}
\usepackage{amssymb}
\usepackage{latexsym}
\usepackage{syntax}
\usepackage{lscape}
\usepackage{stmaryrd}
\usepackage{microtype}
\usepackage{graphicx}
\usepackage{mathpartir}
%\usepackage[pdftex]{graphicx}

%\usetikzlibrary{positioning,calc}

\DeclareMathSymbol{\mlq}{\mathord}{operators}{``}
\DeclareMathSymbol{\mrq}{\mathord}{operators}{`'}
\DeclareMathOperator*{\argmin}{\arg\!\min}
\DeclareMathOperator*{\argmax}{\arg\!\max}


\def\extraVskip{3pt}
\newenvironment{scprooftree}[1]%
  {\gdef\scalefactor{#1}\begin{center}\proofSkipAmount \leavevmode}%
  {\scalebox{\scalefactor}{\DisplayProof}\proofSkipAmount \end{center} }

\makeatletter
\providecommand{\bigsqcap}{%
  \mathop{%
    \mathpalette\@updown\bigsqcup
  }%
}
\newcommand*{\@updown}[2]{%
  \rotatebox[origin=c]{180}{$\m@th#1#2$}%
}
\makeatother

\setlength{\parindent}{0cm}


\begin{document}
\pagenumbering{arabic}


\newcommand{\hasTypePremise}[3]{[#1 : #2]_{#3}}
\newcommand{\hasTypeFormula}[3]{{\llbracket #1 : #2\rrbracket}_{#3}}
\newcommand{\accFor}[1]{\llbracket #1 \rrbracket}

\newcommand{\Heap}{H}

\newcommand{\sfrme}{\ensuremath{\vdash_\texttt{sfrm}}}
\newcommand{\sfrmphi}{\ensuremath{\vdash_\texttt{sfrm}}}
\newcommand{\true}{\ensuremath{\texttt{true}}}
\newcommand{\vnull}{\ensuremath{\texttt{null}}}
\newcommand{\Tint}{\ensuremath{\texttt{int}}}
\newcommand{\xresult}{\ensuremath{\texttt{result}}}
\newcommand{\xthis}{\ensuremath{\texttt{this}}}
\newcommand{\new}{\ensuremath{\texttt{new}~}}
\newcommand{\assert}{\ensuremath{\texttt{assert}~}}
\newcommand{\release}{\ensuremath{\texttt{release}~}}
\newcommand{\return}{\ensuremath{\texttt{return}~}}
\newcommand{\acc}{\ensuremath{\texttt{acc}}}
\newcommand{\fields}{\ensuremath{\texttt{fields}}}
\newcommand{\mpre}{\ensuremath{\texttt{mpre}}}
\newcommand{\mpost}{\ensuremath{\texttt{mpost}}}
\newcommand{\mbody}{\ensuremath{\texttt{mbody}}}
\newcommand{\mparam}{\ensuremath{\texttt{mparam}}}
\newcommand{\mrettype}{\ensuremath{\texttt{mrettype}}}
\newcommand{\mmethod}{\ensuremath{\texttt{mmethod}}}

%\newcommand{\staticFP}[1]{\ensuremath{\texttt{static-footprint}(#1)}}
%\newcommand{\dynamicFP}[3]{\ensuremath{\texttt{footprint}_{#1,#2}(#3)}}

\newcommand\floor[1]{\lfloor#1\rfloor}
\newcommand\ceil[1]{\lceil#1\rceil}
\newcommand{\staticFP}[1]{\ensuremath{\floor{#1}}}
\newcommand{\dynamicFP}[3]{\ensuremath{\floor{#3}_{#1,#2}}}

\newcommand{\rlabel}[1]{\RightLabel{\quad #1}}
\newcommand{\dom}{\ensuremath{\texttt{dom}}}

\newcommand{\class}{\ensuremath{\texttt{class}~}}
\newcommand{\requires}{\ensuremath{\texttt{requires}~}}
\newcommand{\ensures}{\ensuremath{\texttt{ensures}~}}

\newcommand{\grad}[1]{\widetilde{#1}}
\newcommand{\withqm}[1]{?\:*\:#1}

\newcommand{\hoare}[3]{\vdash\{#1\}#2\{#3\}}
\newcommand{\ghoare}[3]{\grad{\vdash}\{#1\}#2\{#3\}}

\newcommand{\hsc}{~\hat{*}~}
\newcommand{\gsc}{~\grad{*}~}
\newcommand{\wo}[2]{#1\textbf{[w/o~}#2\textbf{]}} 

\section{Syntax}

\begin{align*}
\\ &program    	&&::= \overline{cls}~\overline{s}
\\ &cls    		&&::= \class C~\{\overline{field}~\overline{method}\}
\\ &field    	&&::= T~f;
\\ &method		&&::= T~m(T~x)~contract~\{\overline{s}\}
\\ &contract	&&::= \requires \phi;~\ensures \phi;
\\ &T			&&::= \Tint ~|~ C
\\ &s			&&::= x.f := y;
				  ~|~ x := e; 
				  ~|~ x := \new C; 
				  ~|~ x := y.m(z);
\\ & &&
				  ~|~ \return x; 
				  ~|~ \assert \phi; 
				  ~|~ \release \phi;
				  ~|~ T~x;
\\ &\phi		&&::= \true
				  ~|~ e = e
				  ~|~ e \neq e
				  ~|~ \acc(e.f)
				  ~|~ \phi * \phi
\\ &e			&&::= v
				  ~|~ x
				  ~|~ e.f
\\ &x			&&::= \xthis ~|~ \xresult ~|~ \langle other \rangle
\\ &v			&&::= o ~|~ n ~|~ \vnull
\\ &n			&&\in~~ \mathbb{Z}
\\				  
\\ &H			&&\in~~ (o \rightharpoonup (C,\overline{(f \rightharpoonup v)}))
\\ &\rho		&&\in~~ (x \rightharpoonup v)
\\ &\Gamma		&&\in~~ (x \rightharpoonup T)
\\ &A_s			&&::= \overline{(e, f)}
\\ &A_d			&&::= \overline{(o, f)}
\\ &S			&&::= (\rho, A_d, \overline{s}) \cdot S ~|~ nil
\end{align*}

\newcommand{\OK}{~\texttt{OK}}
\newcommand{\OKinC}{~\texttt{OK in}~C}
\section{Assumptions}
All the rules in the following sections are implicitly parameterized over a $program p$ that is well-formed.

\subsubsection{Well-formed program ($program \OK$)}
\begin{mathpar}
\inferrule* [Right=OKProgram]
{
\overline{cls_i \OK}
}
{(\overline{cls_i}~\overline{s}) \OK}
\end{mathpar}

\subsubsection{Well-formed class ($cls \OK$)}
\begin{mathpar}
\inferrule* [Right=OKClass]
{
\text{unique $field$-names} \\
\text{unique $method$-names} \\
\overline{method_i \OKinC}
}
{(\class C~\{\overline{field_i}~\overline{method_i}\}) \OK}
\end{mathpar}

\subsubsection{Well-formed method ($method \OKinC$)}
\begin{mathpar}
\inferrule* [Right=OKMethod]
{
FV(\phi_1) \subseteq \{ x, \xthis \} \\
FV(\phi_2) \subseteq \{ x, \xthis, \xresult \} \\
x : T_x, \xthis : C, \xresult : T_m \hoare {\phi_1} {\overline{s}} {\phi_2} \\
\emptyset \sfrmphi {\phi_1} \\
\emptyset \sfrmphi {\phi_2} \\
\overline{\neg writesTo(s_i, x)}
}
{(T_m~m(T_x~x)~\requires \phi_1;~\ensures \phi_2;~\{\overline{s}\}) \OKinC}
\end{mathpar}


\section{Static semantics}
\subsection{Expressions ($A_s \sfrme e$)}
\begin{mathpar}
\inferrule* [Right=WFVar]
{~}
{A \sfrme x}
\end{mathpar}

\begin{mathpar}
\inferrule* [Right=WFValue]
{~}
{A \sfrme v}
\end{mathpar}

\begin{mathpar}
\inferrule* [Right=WFField]
{( x , f ) \in A}
{A \sfrme x.f}
\end{mathpar}



\subsection{Formulas ($A_s \sfrme \phi$)}
% Inductive Semantics.sfrmphi'
\begin{mathpar}
\inferrule* [Right=WFTrue]
{~}
{{A} \sfrmphi {\true}}
\end{mathpar}

\begin{mathpar}
\inferrule* [Right=WFEqual]
{{A} \sfrme {e_1}\\{A} \sfrme {e_2}}
{{A} \sfrmphi {({e_1} = {e_2})}}
\end{mathpar}

\begin{mathpar}
\inferrule* [Right=WFNEqual]
{{A} \sfrme {e_1}\\{A} \sfrme {e_2}}
{{A} \sfrmphi {({e_1} \neq {e_2})}}
\end{mathpar}

\begin{mathpar}
\inferrule* [Right=WFAcc]
{{A} \sfrme {e}}
{{A} \sfrmphi {\acc({e}.{f})}}
\end{mathpar}

\begin{mathpar}
\inferrule* [Right=WFType]
{~}
{{A} \sfrmphi {({x} : {T})}}
\end{mathpar}



\begin{mathpar}
\inferrule* [Right=WFSepOp]
{
A_s \sfrmphi \phi_1 \\ 
A_s \cup \staticFP {\phi_1} \sfrmphi \phi_2
}
{A_s \sfrmphi \phi_1 * \phi_2}
\end{mathpar}

\subsubsection{Implication ($\phi_1 \dot{\implies} \phi_2$)}
Conservative approx. of $\phi_1 \implies \phi_2$.

\subsection{Footprint ($\staticFP {\phi} = A_s$)}
\begin{align*}
 &\staticFP {\true}    		&&= \emptyset
\\ &\staticFP {e_1 = e_2}     	&&= \emptyset
\\ &\staticFP {e_1 \neq e_2}  	&&= \emptyset
\\ &\staticFP {\acc(e.f)} 		&&= \{(e,f)\}
\\ &\staticFP {\phi_1 * \phi_2} 	&&= \staticFP {\phi_1} \cup \staticFP {\phi_2}
\end{align*}

\newcommand{\sType}[3]{#1 \vdash #2 : #3}
\subsection{Type ($\sType {\Gamma} {e} {T}$)}
%\begin{mathpar}
%\inferrule* [Right=STValue]
%{~}
%{\sType {\phi} {\null} {T}}
%\end{mathpar}
\begin{mathpar}
\inferrule* [Right=STValNum]
{~}
{\sType {\Gamma} {n} {\Tint}}
\end{mathpar}
\begin{mathpar}
\inferrule* [Right=STValNull]
{~}
{\sType {\Gamma} {\vnull} {T}}
\end{mathpar}

\begin{mathpar}
\inferrule* [Right=STVar]
{\Gamma(x) = T}
{\sType {\Gamma} {x} {T}}
\end{mathpar}

\begin{mathpar}
\inferrule* [Right=STField]
{ \sType {\Gamma} {e} {C}
\\ \vdash C.f : T}
{\sType {\Gamma} {e.f} {T}}
\end{mathpar}

\subsection{Hoare ($\Gamma \hoare {\phi} {\overline s} {\phi}$)}
\begin{mathpar}
\inferrule* [right=HNewObj]
{ \Gamma(x') = C'  \\  \fields(C') = fs }
{\hoare {Gamma} {p} {x'{\::=\:\new\:}C'} {(\overline{ \acc(x',fs) :: ( x'{\:\neq\:}\vnull :: p ) })}}
\end{mathpar}\hfill

\begin{mathpar}
\inferrule* [right=HFieldAssign]
{\acc(x',f') \in p \\ x'{\:\neq\:}\vnull \in p \\ y'{\:=\:}e' \in p}
{\hoare {Gamma} {p} {x'{\::=\:}f'.y'} {p{\:*\:}x'.f'{\:=\:}y'}}
\end{mathpar}\hfill

\begin{mathpar}
\inferrule* [right=HVarAssign]
{ p' = p[x'/e']  \\ e'{\:=\:}e2' \in p' \\  sfrmphi [] p'  \\ (staticFootprint p') \sfrme e'}
{\hoare {Gamma} {p'} {(sAssign x' e')} {p}}
\end{mathpar}\hfill

\begin{mathpar}
\inferrule* [right=HReturn]
{~}
{\hoare {Gamma} {p} {{\return}x'} {p{\:*\:}\xresult{\:=\:}x'}}
\end{mathpar}\hfill

\begin{mathpar}
\inferrule* [right=HApp]
{ (snd pr) )) Xz' ), \Gamma(y') = C'  \\ y'{\:\neq\:}\vnull \in p \\ p{\:\implies\:}( pp ++ pr ) \\  pp = option_map (phiSubsts ( ( \xthis , y' ) :: Xze' )) (mpre C' m')  \\  pq = option_map (phiSubsts (( ( \xthis , y' ) :: Xze' ){\:*\:}( \xresult , x' ))) (mpost C' m') }
{\hoare {Gamma} {p} {(sCall x' y' m' zs')} {( pq ++ pr )}}
\end{mathpar}\hfill

\begin{mathpar}
\inferrule* [right=HAssert]
{p2 \in p1}
{\hoare {Gamma} {p1} {{\assert}p2} {p1}}
\end{mathpar}\hfill

\begin{mathpar}
\inferrule* [right=HRelease]
{p1{\:\implies\:}( p2 :: pr ) \\  sfrmphi [] pr }
{\hoare {Gamma} {p1} {{\release}p2} {pr}}
\end{mathpar}\hfill



\subsubsection{Notation}
% Inductive Semantics.hoare

\begin{mathpar}
\inferrule* [Right=HNewObj]
{\hat{\phi} \implies \hat{\phi'}\\{x} \not \in {FV({\hat{\phi'}})}\\{\Gamma} \vdash {{x}} : {{C}}\\{\fields({C})} = {{\overline{f}}}}
{{\Gamma} \hoare {\hat{\phi}} {{{x} := \new {C}}} {{\hat{\phi'}} \hsc {{({{x}} \neq {{\vnull}})} \hsc {\overline{\acc({x}, f_i)}}}}}
\end{mathpar}

\begin{mathpar}
\inferrule* [Right=HFieldAssign]
{\hat{\phi} \implies {{\acc({{x}}.{f})} * {\hat{\phi'}}}\\{\Gamma} \vdash {{x}} : {{C}}\\{\Gamma} \vdash {{y}} : {T}\\\vdash {C}.{f} : {T}}
{{\Gamma} \hoare {\hat{\phi}} {{{x}.{f} := {y}}} {\hat{\phi'} \hsc {{\acc({{x}}.{f})} \hsc {{({{x}} \neq {{\vnull}})} \hsc {{({{{x}}.{f}} = {{y}})}}}}}}
\end{mathpar}

\begin{mathpar}
\inferrule* [Right=HVarAssign]
{\hat{\phi} \implies \hat{\phi'}\\{x} \not \in {FV(\hat{\phi'})}\\{x} \not \in {FV({e})}\\{\Gamma} \vdash {{x}} : {T}\\{\Gamma} \vdash {e} : {T}\\{\accFor {{e}}} \subseteq {\hat{\phi'}}}
{{\Gamma} \hoare {\hat{\phi}} {{{x} := {e}}} {\hat{\phi'} \hsc {{({{x}} = {e})}}}}
\end{mathpar}

\begin{mathpar}
\inferrule* [Right=HReturn]
{\hat{\phi} \implies \hat{\phi'}\\{\xresult} \not \in {FV(\hat{\phi'})}\\{\Gamma} \vdash {{x}} : {T}\\{\Gamma} \vdash {{\xresult}} : {T}}
{{\Gamma} \hoare {\hat{\phi}} {{\return {x}}} {\hat{\phi'} \hsc {{({{\xresult}} = {{x}})}}}}
\end{mathpar}

\begin{mathpar}
\inferrule* [Right=HApp]
{{\Gamma} \vdash {{y}} : {{C}}\\{\mmethod({C}, {m})} = {{{T_r}~{m}({T_p}~{z})~{\requires \hat{\phi_{pre}};~\ensures \hat{\phi_{post}};}~\{ {\_} \}}}\\{\Gamma} \vdash {{x}} : {T_r}\\{\Gamma} \vdash {{z'}} : {T_p}\\\hat{\phi} \implies {{({{y}} \neq {{\vnull}})} * {\hat{\phi_p} * \hat{\phi'}}}\\{x} \not \in {FV(\hat{\phi'})}\\x \neq y \wedge x \neq z'\\\hat{\phi_p} = {\hat{\phi_{pre}}[{y}, {z'} / {\xthis}, {{z}}]}\\\hat{\phi_q} = {\hat{\phi_{post}}[{y}, {z'}, {x} / {\xthis}, {{z}}, {\xresult}]}}
{{\Gamma} \hoare {\hat{\phi}} {{{x} := {y}.{m}({z'})}} {\hat{\phi'} * \hat{\phi_q}}}
\end{mathpar}

\begin{mathpar}
\inferrule* [Right=HAssert]
{\hat{\phi} \implies {\phi'}}
{{\Gamma} \hoare {\hat{\phi}} {{\assert {\phi'}}} {\hat{\phi}}}
\end{mathpar}

\begin{mathpar}
\inferrule* [Right=HRelease]
{\hat{\phi} \implies {{\phi_r} * \hat{\phi'}}}
{{\Gamma} \hoare {\hat{\phi}} {{\release {\phi_r}}} {\hat{\phi'}}}
\end{mathpar}

\begin{mathpar}
\inferrule* [Right=HDeclare]
{{x} \not\in \dom({\Gamma})\\{{\Gamma}, {x} : {T}} \hoare {\hat{\phi} \hsc {({{x}} = {{\texttt{defaultValue}({T})}})}} {\overline{s}} {\hat{\phi'}}}
{{\Gamma} \hoare {\hat{\phi}} {{{T}~{x}}; {\overline{s}}} {\hat{\phi'}}}
\end{mathpar}

\begin{mathpar}
\inferrule* [Right=HSec]
{{\Gamma} \hoare {\hat{\phi_p}} {\overline{s_1}} {\hat{\phi_q}}\\{\Gamma} \hoare {\hat{\phi_q}} {\overline{s_2}} {\hat{\phi_r}}}
{{\Gamma} \hoare {\hat{\phi_p}} {\overline{s_1}; \overline{s_2}} {\hat{\phi_r}}}
\end{mathpar}



\subsubsection{Deterministic}
% Inductive Semantics.hoare

\begin{mathpar}
\inferrule* [Right=HNewObj]
{{\wo {\hat{\phi}} {x}} = \hat{\phi'}\\
    {\Gamma} \vdash {{x}} : {{C}}\\{\fields({C})} = {{\overline{f}}}}
{{\Gamma} \hoare {\hat{\phi}} {{{x} := \new {C}}} {{\hat{\phi'}} \hsc {{({{x}} \neq {{\vnull}})} \hsc {\overline{\acc({x}, f_i)}}}}}
\end{mathpar}

\begin{mathpar}
\inferrule* [Right=HFieldAssign]
{{\wo {\hat{\phi}} {\acc(x.f)}} = \hat{\phi'}\\
    \hat{\phi} \implies {{\acc({{x}}.{f})}}\\{\Gamma} \vdash {{x}} : {{C}}\\{\Gamma} \vdash {{y}} : {T}\\\vdash {C}.{f} : {T}}
{{\Gamma} \hoare {\hat{\phi}} {{{x}.{f} := {y}}} {\hat{\phi'} \hsc {{\acc({{x}}.{f})} \hsc {{({{x}} \neq {{\vnull}})} \hsc {{({{{x}}.{f}} = {{y}})}}}}}}
\end{mathpar}

\begin{mathpar}
\inferrule* [Right=HVarAssign]
{{\wo {\hat{\phi}} {x}} = \hat{\phi'}\\
    {x} \not \in {FV({e})}\\{\Gamma} \vdash {{x}} : {T}\\{\Gamma} \vdash {e} : {T}\\{\hat{\phi'}} \implies {\accFor {{e}}}}
{{\Gamma} \hoare {\hat{\phi}} {{{x} := {e}}} {\hat{\phi'} \hsc {{({{x}} = {e})}}}}
\end{mathpar}

Have $\hsc$ take care of necessary congruent rewriting of $e$ in order to preserve self-framing!

\begin{mathpar}
\inferrule* [Right=HReturn]
{{\wo {\hat{\phi}} {\xresult}} = \hat{\phi'}\\
    {\Gamma} \vdash {{x}} : {T}\\{\Gamma} \vdash {{\xresult}} : {T}}
{{\Gamma} \hoare {\hat{\phi}} {{\return {x}}} {\hat{\phi'} \hsc {{({{\xresult}} = {{x}})}}}}
\end{mathpar}

\begin{mathpar}
\inferrule* [Right=HApp]
{\wo {\wo {\hat{\phi}} {x}} {\staticFP{\hat{\phi_p}}} = \hat{\phi'}\\
    {\Gamma} \vdash {{y}} : {{C}}\\{\mmethod({C}, {m})} = {{{T_r}~{m}({T_p}~{z})~{\requires \hat{\phi_{pre}};~\ensures \hat{\phi_{post}};}~\{ {\_} \}}}\\{\Gamma} \vdash {{x}} : {T_r}\\{\Gamma} \vdash {{z'}} : {T_p}\\\hat{\phi} \implies {{\hat{\phi_p}} \hsc {({{y}} \neq {{\vnull}})}}\\x \neq y \wedge x \neq z'\\\hat{\phi_p} = {\hat{\phi_{pre}}[{y}, {z'} / {\xthis}, {{z}}]}\\\hat{\phi_q} = {\hat{\phi_{post}}[{y}, {z'}, {x} / {\xthis}, {{z}}, {\xresult}]}}
{{\Gamma} \hoare {\hat{\phi}} {{{x} := {y}.{m}({z'})}} {\hat{\phi'} \hsc \hat{\phi_q}}}
\end{mathpar}

\begin{mathpar}
\inferrule* [Right=HAssertBAD (grad lifting non-trivial!)]
{\hat{\phi} \implies {\phi_a}}
{{\Gamma} \hoare {\hat{\phi}} {{\assert {\phi_a}}} {\hat{\phi}}}
\end{mathpar}

\begin{mathpar}
    \inferrule* [Right=HAssert]
    {{\wo {\hat{\phi}} {\staticFP{\phi_a}}} = \hat{\phi'}\\
        \hat{\phi} \implies {\phi_a}}
    {{\Gamma} \hoare {\hat{\phi}} {{\assert {\phi_a}}} {\hat{\phi'} \hsc \phi_a}}
\end{mathpar}

\begin{mathpar}
\inferrule* [Right=HRelease]
{{\wo {\hat{\phi}} {\staticFP{\phi_r}}} = \hat{\phi'}\\
    \hat{\phi} \implies {{\phi_r}}}
{{\Gamma} \hoare {\hat{\phi}} {{\release {\phi_r}}} {\hat{\phi'}}}
\end{mathpar}

\begin{mathpar}
\inferrule* [Right=HDeclare]
{{x} \not\in \dom({\Gamma})\\{{\Gamma}, {x} : {T}} \hoare {\hat{\phi} \hsc {({{x}} = {{\texttt{defaultValue}({T})}})}} {\overline{s}} {\hat{\phi'}}}
{{\Gamma} \hoare {\hat{\phi}} {{{T}~{x}}; {\overline{s}}} {\hat{\phi'}}}
\end{mathpar}

\begin{mathpar}
\inferrule* [Right=HSec]
{{\Gamma} \hoare {\hat{\phi_p}} {s_1} {\hat{\phi_q}}\\{\Gamma} \hoare {\hat{\phi_q}} {s_2} {\hat{\phi_r}}}
{{\Gamma} \hoare {\hat{\phi_p}} {{s_1}; {s_2}} {\hat{\phi_r}}}
\end{mathpar}



\subsubsection{Gradual}
% Inductive Semantics.hoare

\begin{mathpar}
\inferrule* [Right=GHNewObj]
{{\wo {\grad{\phi}} {x}} = \grad{\phi'}\\
    {\Gamma} \vdash {{x}} : {{C}}\\{\fields({C})} = {{\overline{f}}}}
{\Gamma~ \ghoare {\grad{\phi}} {{{x} := \new {C}}} {{\grad{\phi'}} \gsc {{({{x}} \neq {{\vnull}})} \gsc {\overline{\acc({x}, f_i)}}}}}
\end{mathpar}

\begin{mathpar}
\inferrule* [Right=GHFieldAssign]
{{\wo {\grad{\phi}} {\acc(x.f)}} = \grad{\phi'}\\
    \grad{\phi} \grad{\implies} {{\acc({{x}}.{f})}}\\{\Gamma} \vdash {{x}} : {{C}}\\{\Gamma} \vdash {{y}} : {T}\\\vdash {C}.{f} : {T}}
{\Gamma~ \ghoare {\grad{\phi}} {{{x}.{f} := {y}}} {\grad{\phi'} \gsc {{\acc({{x}}.{f})} \gsc {{({{x}} \neq {{\vnull}})} \gsc {{({{{x}}.{f}} = {{y}})}}}}}}
\end{mathpar}

\begin{mathpar}
\inferrule* [Right=GHVarAssign]
{{\wo {\grad{\phi}} {x}} = \grad{\phi'}\\
    {x} \not \in {FV({e})}\\{\Gamma} \vdash {{x}} : {T}\\{\Gamma} \vdash {e} : {T}\\{\grad{\phi'}} \grad{\implies} {\accFor {{e}}}}
{\Gamma~ \ghoare {\grad{\phi}} {{{x} := {e}}} {\grad{\phi'} \gsc {{({{x}} = {e})}}}}
\end{mathpar}

Let $\gsc$ behave like $\hsc$ if first operand is static - otherwise its regular concatenation.

\begin{mathpar}
\inferrule* [Right=GHReturn]
{{\wo {\grad{\phi}} {\xresult}} = \grad{\phi'}\\
    {\Gamma} \vdash {{x}} : {T}\\{\Gamma} \vdash {{\xresult}} : {T}}
{\Gamma~ \ghoare {\grad{\phi}} {{\return {x}}} {\grad{\phi'} \gsc {{({{\xresult}} = {{x}})}}}}
\end{mathpar}

\begin{mathpar}
\inferrule* [Right=GHApp]
{\wo {\wo {\grad{\phi}} {x}} {\staticFP{\grad{\phi_p}}_{\Gamma, y, z'}} = \grad{\phi'}\\
    {\Gamma} \vdash {{y}} : {{C}}\\{\mmethod({C}, {m})} = {{{T_r}~{m}({T_p}~{z})~{\requires \grad{\phi_{pre}};~\ensures \grad{\phi_{post}};}~\{ {\_} \}}}\\{\Gamma} \vdash {{x}} : {T_r}\\{\Gamma} \vdash {{z'}} : {T_p}\\\grad{\phi} \grad{\implies} {{\grad{\phi_p}} \gsc {({{y}} \neq {{\vnull}})}}\\x \neq y \wedge x \neq z'\\\grad{\phi_p} = {\grad{\phi_{pre}}[{y}, {z'} / {\xthis}, {{z}}]}\\\grad{\phi_q} = {\grad{\phi_{post}}[{y}, {z'}, {x} / {\xthis}, {{z}}, {\xresult}]}}
{\Gamma~ \ghoare {\grad{\phi}} {{{x} := {y}.{m}({z'})}} {\grad{\phi'} \gsc \grad{\phi_q}}}
\end{mathpar}

\begin{mathpar}
\inferrule* [Right=GHAssert]
{\grad{\phi} \grad{\implies} {\phi'}}
{\Gamma~ \ghoare {\grad{\phi}} {{\assert {\phi'}}} {\grad{\phi}}}
\end{mathpar}

\begin{mathpar}
\inferrule* [Right=GHRelease]
{{\wo {\grad{\phi}} {\staticFP{\phi_r}}} = \grad{\phi'}\\
    \grad{\phi} \grad{\implies} {{\phi_r}}}
{\Gamma~ \ghoare {\grad{\phi}} {{\release {\phi_r}}} {\grad{\phi'}}}
\end{mathpar}

\begin{mathpar}
\inferrule* [Right=GHDeclare]
{{x} \not\in \dom({\Gamma})\\{{\Gamma}, {x} : {T}} \hoare {\grad{\phi} \gsc {({{x}} = {{\texttt{defaultValue}({T})}})}} {\overline{s}} {\grad{\phi'}}}
{\Gamma~ \ghoare {\grad{\phi}} {{{T}~{x}}; {\overline{s}}} {\grad{\phi'}}}
\end{mathpar}

\begin{mathpar}
\inferrule* [Right=GHSec]
{\Gamma~ \ghoare {\grad{\phi_p}} {s_1} {\grad{\phi_q}}\\\Gamma~ \ghoare {\grad{\phi_q}} {s_2} {\grad{\phi_r}}}
{\Gamma~ \ghoare {\grad{\phi_p}} {{s_1}; {s_2}} {\grad{\phi_r}}}
\end{mathpar}



\section{Dynamic semantics}
\newcommand{\evalex}[4]{#1,#2 \vdash #3 \Downarrow #4}
\newcommand{\evale}[2]{H,\rho \vdash #1 \Downarrow #2}
\subsection{Expressions ($\evale {e} {v}$)}

\begin{mathpar}
\inferrule* [Right=EEVar]
{~}
{\evale {x} {\rho(x)}}
\end{mathpar}

\begin{mathpar}
\inferrule* [Right=EEValue]
{~}
{\evale {v} {v}}
\end{mathpar}

\begin{mathpar}
\inferrule* [Right=EEAcc]
{\evale {e} {o}}
{\evale {e.f} {H(o)(f)}}
\end{mathpar}

\newcommand{\evalphix}[4]{#1,#2,#3 \vDash #4}
\newcommand{\evalphi}{\evalphix H \rho A}
\newcommand{\valphi}[1]{\llbracket #1 \rrbracket}
\subsection{Formulas ($\evalphi \phi$)}
% Inductive Semantics.evalphi'
\begin{mathpar}
\inferrule* [Right=EATrue]
{~}
{\evalphix {H} {\rho} {A} {\phiTrue}}
\end{mathpar}

\begin{mathpar}
\inferrule* [Right=EAEqual]
{\evalex {H} {\rho} {e_1} {v_1}\\\evalex {H} {\rho} {e_2} {v_2}\\{v_1} = {v_2}}
{\evalphix {H} {\rho} {A} {\phiEq {${e_1}$} {${e_2}$}}}
\end{mathpar}

\begin{mathpar}
\inferrule* [Right=EANEqual]
{\evalex {H} {\rho} {e_1} {v_1}\\\evalex {H} {\rho} {e_2} {v_2}\\{v_1} \neq {v_2}}
{\evalphix {H} {\rho} {A} {\phiNeq {${e_1}$} {${e_2}$}}}
\end{mathpar}

\begin{mathpar}
\inferrule* [Right=EAAcc]
{\evalex {H} {\rho} {e} {{o}}\\\evalex {H} {\rho} {\edot{${e}$}{${f}$}} {v}\\{({o}, {f})} \in {A}}
{\evalphix {H} {\rho} {A} {\phiAcc {${e}$} {${f}$}}}
\end{mathpar}



\begin{mathpar}
\inferrule* [Right=EASepOp]
{
A_1 = A \backslash A_2 \\
\evalphix H \rho {A_1} {\phi_1} \\
\evalphix H \rho {A_2} {\phi_2}
}
{\evalphi {\phi_1 * \phi_2}}
\end{mathpar}

We give a denotational semantics of formulas as $\valphi {\phi} = \{~ (H,\rho,A) ~|~ \evalphi {\phi} ~\}$

Note: $\phi \text{ satisfiable} \iff \valphi {\phi} \neq \emptyset$

\subsubsection{Implication ($\phi_1 \implies \phi_2$)}
%\begin{equation*}
%\phi_1 \implies \phi_2
%\quad\quad \iff \quad\quad
%\valphi {\phi_1} \subseteq \valphi {\phi_2}
%\end{equation*}

\begin{equation*}
\phi_1 \implies \phi_2
\quad\quad \iff \quad\quad
\forall H, \rho, A: \evalphi \phi_1 \implies \evalphi \phi_2
\end{equation*}
Drawn from def. of entailment in ``A Formal Semantics for Isorecursive and Equirecursive State Abstractions''.

\subsubsection{Implying inequality}
\begin{align*}
\\   & \phi * (e_1 = e_1) * (e_2 = e_2) \implies (e_1 \neq e_2)
\\ =~& \forall H, \rho, A:~ \evalphi \phi * (e_1 = e_1) * (e_2 = e_2) \implies \evalphi (e_1 \neq e_2)
\\ =~& \forall H, \rho, A:~  (\exists v_1, v_2:~ \evale {e_1} {v_1} \wedge \evale {e_2} {v_2} \wedge \evalphi \phi) \implies (\exists v_1, v_2:~ \evale {e_1} {v_1} \wedge \evale {e_2} {v_2} \wedge (v_1 \neq v2))
\\ =~& \forall H, \rho, A, v_1, v_2:~  (\evale {e_1} {v_1} \wedge \evale {e_2} {v_2} \wedge \evalphi \phi) \implies (\exists v_1, v_2:~ \evale {e_1} {v_1} \wedge \evale {e_2} {v_2} \wedge (v_1 \neq v2))
\\ =~& \forall H, \rho, A, v_1, v_2:~  (\evale {e_1} {v_1} \wedge \evale {e_2} {v_2} \wedge \evalphi \phi) \implies (v_1 \neq v2)
\\ =~& \forall H, \rho, A, v_1, v_2:~  \neg (\evale {e_1} {v_1} \wedge \evale {e_2} {v_2} \wedge \evalphi \phi) \vee (v_1 \neq v2)
\\ =~& \forall H, \rho, A, v_1, v_2:~  \neg (\evale {e_1} {v_1} \wedge \evale {e_2} {v_2} \wedge \evalphi \phi \wedge (v_1 = v2))
\\ =~& \forall H, \rho, A:~ \neg (\exists v_1, v_2:~ \evale {e_1} {v_1} \wedge \evale {e_2} {v_2} \wedge \evalphi \phi \wedge (v_1 = v2))
\\ =~& \forall H, \rho, A:~ \neg (\evalphi \phi \wedge \evalphi (e_1 = e_2))
\\ =~& \forall H, \rho, A:~ \neg \evalphi \phi * (e_1 = e_2)
\\ =~& \neg \text{sat~}(\phi * (e_1 = e_2))
\end{align*}

\subsection{Footprint ($\dynamicFP {H} {\rho} {\phi} = A_d$)}
\begin{align*}
 &\dynamicFP {H} {\rho} {\true}    		&&= \emptyset
\\ &\dynamicFP {H} {\rho} {e_1 = e_2}     	&&= \emptyset
\\ &\dynamicFP {H} {\rho} {e_1 \neq e_2}  	&&= \emptyset
\\ &\dynamicFP {H} {\rho} {\acc(x.f)} 		&&= \{(o,f)\} \text{ where } \evale x o
\\ &\dynamicFP {H} {\rho} {\phi_1 * \phi_2} &&= \dynamicFP {H} {\rho} {\phi_1} \cup \dynamicFP {H} {\rho} {\phi_2}
\end{align*}

\newcommand{\dType}[4]{#1, #2 \vdash #3 : #4}

\newcommand{\sstepGeneric}[5]{({#1}, {#2}) \rightarrow^{#3} ({#4}, {#5})}
\newcommand{\sstep}[4]{\sstepGeneric {#1} {#2} {} {#3} {#4}}
\newcommand{\sstepM}[4]{\sstepGeneric {#1} {#2} * {#3} {#4}} 
\newcommand{\sstepWS}[4]{\sstepGeneric {#1} {{#2} \cdot S} {} {#3} {{#4} \cdot S}}
\newcommand{\sstepWSX}[8]{\sstepGeneric {#1} {({#2},{#3},{#4}) \cdot S} {} {#5} {({#6},{#7},{#8}) \cdot S}}

\newcommand{\Tfs}{\overline{T}~\overline{f}}
\subsection{Small-step ($\sstep H S H S$)}
\begin{mathpar}
\inferrule* [Right=ESFieldAssign]
{\evalex {\Heap} {\rho} {x} {o}\\\evalex {\Heap} {\rho} {y} {v_y}\\\textcolor{gray}{( o , f ) \in A}\\ \Heap' = \Heap[o{\:\mapsto\:}[f{\:\mapsto\:}v_y]] }
{{( \Heap , ( \rho ,\textcolor{gray}{ A }, x.f{\::=\:}y; \overline{s} ) \cdot S )} \rightarrow {( \Heap' , ( \rho ,\textcolor{gray}{ A }, \overline{s} ) \cdot S )}}
\end{mathpar}

\begin{mathpar}
\inferrule* [Right=ESVarAssign]
{\evalex {\Heap} {\rho} {e} {v}\\ \rho' = \rho[x{\:\mapsto\:}v] }
{{( \Heap , ( \rho ,\textcolor{gray}{ A }, x{\::=\:}e; \overline{s} ) \cdot S )} \rightarrow {( \Heap , ( \rho' ,\textcolor{gray}{ A }, \overline{s} ) \cdot S )}}
\end{mathpar}

\begin{mathpar}
\inferrule* [Right=ESNewObj]
{ \Heap(o) = \bot \\ \fields(C) = f \\ \rho' = \rho[x{\:\mapsto\:}o] \\\textcolor{gray}{ A' = A * \overline{ ( o , f_i ) } }\\ \Heap' = \Heap[o{\:\mapsto\:\new\:}C] }
{{( \Heap , ( \rho ,\textcolor{gray}{ A }, x{\::=\:\new\:}C; \overline{s} ) \cdot S )} \rightarrow {( \Heap' , ( \rho' ,\textcolor{gray}{ A' }, \overline{s} ) \cdot S )}}
\end{mathpar}

\begin{mathpar}
\inferrule* [Right=ESReturn]
{\evalex {\Heap} {\rho} {x} {v_x}\\ \rho' = \rho[\xresult{\:\mapsto\:}v_x] }
{{( \Heap , ( \rho ,\textcolor{gray}{ A }, {\return}x; \overline{s} ) \cdot S )} \rightarrow {( \Heap , ( \rho' ,\textcolor{gray}{ A }, \overline{s} ) \cdot S )}}
\end{mathpar}

\begin{mathpar}
\inferrule* [Right=ESApp]
{\evalex {\Heap} {\rho} {y} {o}\\\evalex {\Heap} {\rho} {z} {v}\\ \Heap(o) = ( C , c ) \\ \mbody(C,m) = \overline{r} \\ \mparam(C,m) = ( T , w ) \\\textcolor{gray}{ \mpre(C,m) = \phi }\\ \rho' = [\xthis{\:\mapsto\:}o,w{\:\mapsto\:}v] \\\textcolor{gray}{\evalphix {\Heap} {\rho'} {A} {\phi}}\\\textcolor{gray}{ A' = \texttt{footprint}_{\Heap,\rho'}(\phi) }}
{{( \Heap , ( \rho ,\textcolor{gray}{ A }, x{\::=\:}y.m(z); \overline{s} ) \cdot S )} \rightarrow {( \Heap , ( \rho' ,\textcolor{gray}{ A' }, \overline{r} ) * ( \rho ,\textcolor{gray}{ A{\:\backslash\:}A' }, x{\::=\:}y.m(z); \overline{s} ) \cdot S )}}
\end{mathpar}

\begin{mathpar}
\inferrule* [Right=ESAppFinish]
{\textcolor{gray}{ \mpost(C,m) = \phi }\\\textcolor{gray}{\evalphix {\Heap} {\rho'} {A'} {\phi}}\\\textcolor{gray}{ A'' = \texttt{footprint}_{\Heap,\rho'}(\phi) }\\\evalex {\Heap} {\rho'} {\xresult} {v_r}}
{{( \Heap , ( \rho' ,\textcolor{gray}{ A' }, \emptyset ) * ( \rho ,\textcolor{gray}{ A }, x{\::=\:}y.m(z); \overline{s} ) \cdot S )} \rightarrow {( \Heap , ( \rho[x{\:\mapsto\:}v_r] ,\textcolor{gray}{ A * A'' }, \overline{s} ) \cdot S )}}
\end{mathpar}

\begin{mathpar}
\inferrule* [Right=ESAssert]
{\textcolor{gray}{\evalphix {\Heap} {\rho} {A} {\phi}}}
{{( \Heap , ( \rho ,\textcolor{gray}{ A }, {\assert}\phi; \overline{s} ) \cdot S )} \rightarrow {( \Heap , ( \rho ,\textcolor{gray}{ A }, \overline{s} ) \cdot S )}}
\end{mathpar}

\begin{mathpar}
\inferrule* [Right=ESRelease]
{\textcolor{gray}{\evalphix {\Heap} {\rho} {A} {\phi}}\\\textcolor{gray}{ A' = A{\:\backslash\:}\texttt{footprint}_{\Heap,\rho}(\phi) }}
{{( \Heap , ( \rho ,\textcolor{gray}{ A }, {\release}\phi; \overline{s} ) \cdot S )} \rightarrow {( \Heap , ( \rho ,\textcolor{gray}{ A' }, \overline{s} ) \cdot S )}}
\end{mathpar}

\begin{mathpar}
\inferrule* [Right=ESDeclare]
{ \rho' = \rho[x{\:\mapsto\:}\texttt{defaultValue}(T)] }
{{( \Heap , ( \rho ,\textcolor{gray}{ A }, T{\:}x; \overline{s} ) \cdot S )} \rightarrow {( \Heap , ( \rho' ,\textcolor{gray}{ A }, \overline{s} ) \cdot S )}}
\end{mathpar}


%% Inductive Semantics.dynSem
\begin{mathpar}
\inferrule* [Right=ESFieldAssign]
{\evalex {H} {\rho} {\ex{${x}$}} {{o}}\\\evalex {H} {\rho} {\ex{${y}$}} {v_y}\\{({o}, {f})} \in {A}\\{H'} = {{H}[{o} \mapsto [{f} \mapsto {v_y}]]}}
{{({H}, {{({{\rho}, {A}}, {{\sMemberSet {${x}$} {${f}$} {${y}$}} {\overline{s}}})} \cdot {S}})} \rightarrow {({H'}, {{({{\rho}, {A}}, {\overline{s}})} \cdot {S}})}}
\end{mathpar}

\begin{mathpar}
\inferrule* [Right=ESVarAssign]
{\evalex {H} {\rho} {e} {v}\\{\rho'} = {{\rho}[{x} \mapsto {v}]}}
{{({H}, {{({{\rho}, {A}}, {{\sAssign {${x}$} {${e}$}} {\overline{s}}})} \cdot {S}})} \rightarrow {({H}, {{({{\rho'}, {A}}, {\overline{s}})} \cdot {S}})}}
\end{mathpar}

\begin{mathpar}
\inferrule* [Right=ESNewObj]
{{o} \not\in \dom({H})\\{\fields({C})} = {{Tfs}}\\{\rho'} = {{\rho}[{x} \mapsto {{o}}]}\\{A'} = {{A} * {@map(prod(T, f), prod(o, f), fun(cf', :, prod, T, f, =>, @pair, o, f, o, @snd(T, f, cf')), Tfs)}}\\{H'} = {{H}[{o} \mapsto [\overline{f \mapsto \texttt{defaultValue}(T)}]]}}
{{({H}, {{({{\rho}, {A}}, {{\sAlloc {${x}$} {${C}$}} {\overline{s}}})} \cdot {S}})} \rightarrow {({H'}, {{({{\rho'}, {A'}}, {\overline{s}})} \cdot {S}})}}
\end{mathpar}

\begin{mathpar}
\inferrule* [Right=ESReturn]
{\evalex {H} {\rho} {\ex{${x}$}} {v_x}\\{\rho'} = {{\rho}[{\xresult} \mapsto {v_x}]}}
{{({H}, {{({{\rho}, {A}}, {{\sReturn {${x}$}} {\overline{s}}})} \cdot {S}})} \rightarrow {({H}, {{({{\rho'}, {A}}, {\overline{s}})} \cdot {S}})}}
\end{mathpar}

\begin{mathpar}
\inferrule* [Right=ESApp]
{\evalex {H} {\rho} {\ex{${y}$}} {{o}}\\\evalex {H} {\rho} {\ex{${z}$}} {v}\\{H(o)} = {{({C}, {\_})}}\\{\mmethod({C}, {m})} = {{\method {${T_r}$} {${m}$} {${T}$} {${w}$} {${\requires {phi};~\ensures {\_};}$} {${\overline{r}}$}}}\\{\rho'} = {[{\xresult} \mapsto {\texttt{defaultValue}({T_r})}, {\xthis} \mapsto {{o}}, {{w}} \mapsto {v}]}\\\evalphix {H} {\rho'} {A} {phi}\\{A'} = {\dynamicFP {H} {\rho'} {phi}}}
{{({H}, {{({{\rho}, {A}}, {{\sCall {${x}$} {${y}$} {${m}$} {${z}$}} {\overline{s}}})} \cdot {S}})} \rightarrow {({H}, {{({{\rho'}, {A'}}, {\overline{r}})} \cdot {{({{\rho}, {{A} \backslash {A'}}}, {{\sCall {${x}$} {${y}$} {${m}$} {${z}$}} {\overline{s}}})} \cdot {S}}})}}
\end{mathpar}

\begin{mathpar}
\inferrule* [Right=ESAppFinish]
{\evalex {H} {\rho} {\ex{${y}$}} {{o}}\\{H(o)} = {{({C}, {\_})}}\\{\mpost({C}, {m})} = {{\phi}}\\\evalphix {H} {\rho'} {A'} {\phi}\\{A''} = {\dynamicFP {H} {\rho'} {\phi}}\\\evalex {H} {\rho'} {\ex{${\xresult}$}} {v_r}}
{{({H}, {{({{\rho'}, {A'}}, {\emptyset})} \cdot {{({{\rho}, {A}}, {{\sCall {${x}$} {${y}$} {${m}$} {${z}$}} {\overline{s}}})} \cdot {S}}})} \rightarrow {({H}, {{({{{\rho}[{x} \mapsto {v_r}]}, {{A} * {A''}}}, {\overline{s}})} \cdot {S}})}}
\end{mathpar}

\begin{mathpar}
\inferrule* [Right=ESAssert]
{\evalphix {H} {\rho} {A} {\phi}}
{{({H}, {{({{\rho}, {A}}, {{\sAssert {${\phi}$}} {\overline{s}}})} \cdot {S}})} \rightarrow {({H}, {{({{\rho}, {A}}, {\overline{s}})} \cdot {S}})}}
\end{mathpar}

\begin{mathpar}
\inferrule* [Right=ESRelease]
{\evalphix {H} {\rho} {A} {\phi}\\{A'} = {{A} \backslash {\dynamicFP {H} {\rho} {\phi}}}}
{{({H}, {{({{\rho}, {A}}, {{\sRelease {${\phi}$}} {\overline{s}}})} \cdot {S}})} \rightarrow {({H}, {{({{\rho}, {A'}}, {\overline{s}})} \cdot {S}})}}
\end{mathpar}

\begin{mathpar}
\inferrule* [Right=ESDeclare]
{{\rho'} = {{\rho}[{x} \mapsto {\texttt{defaultValue}({T})}]}}
{{({H}, {{({{\rho}, {A}}, {{\sDeclare {${T}$} {${x}$}} {\overline{s}}})} \cdot {S}})} \rightarrow {({H}, {{({{\rho'}, {A}}, {\overline{s}})} \cdot {S}})}}
\end{mathpar}

\begin{mathpar}
\inferrule* [Right=ESHold]
{\evalphix {H} {\rho} {A} {\phi}\\{A'} = {\dynamicFP {H} {\rho} {\phi}}}
{{({H}, {{({{\rho}, {A}}, {{\sHold {${\phi}$} {${\overline{s'}}$}} {\overline{s}}})} \cdot {S}})} \rightarrow {({H}, {{({{\rho}, {{A} \backslash {A'}}}, {\overline{s'}})} \cdot {{({{\rho}, {A'}}, {{\sHold {${\phi}$} {${\overline{s'}}$}} {\overline{s}}})} \cdot {S}}})}}
\end{mathpar}

\begin{mathpar}
\inferrule* [Right=ESHoldFinish]
{~}
{{({H}, {{({{\rho'}, {A'}}, {\emptyset})} \cdot {{({{\rho}, {A}}, {{\sHold {${\phi}$} {${\overline{s'}}$}} {\overline{s}}})} \cdot {S}}})} \rightarrow {({H}, {{({{\rho'}, {{A} * {A'}}}, {\overline{s}})} \cdot {S}})}}
\end{mathpar}



\section{Gradualization}
\subsection{Syntax}
\subsubsection{Gradual formula}
\begin{align*}
&\grad{\phi} \quad ::= \quad \phi ~|~ \withqm{\phi}
\end{align*}

Note: consider $?$ in other positions as ``self-framing delimiter'', but with semantically identical meaning.

As long as $?$ is only legal in the front though: $\phi_1 * \grad{\phi_2}$ propagates the $?$ to the very left in case $\grad{\phi_2}$ contains one.

\subsubsection{Self-framed and satisfiable formula}
\begin{align*}
&\hat{\phi} \quad \in \quad \{~ \phi ~|~ \sfrmphi \phi \wedge \text{sat~} \phi ~\}
\end{align*}

\subsection{Concretization}
\begin{align*}
&\gamma(\hat{\phi}) ~&&= \{~ \hat{\phi} ~\} \\
&\gamma(?\:*\:\phi') ~&&= \{~ \hat{\phi} ~|~ \hat{\phi} \implies \phi' ~\} \text{~~if $\phi'$ satisfiable} \\
&\gamma(\phi) \text{ undefined otherwise} \\
~\\
&\grad{\phi_1} \sqsubseteq \grad{\phi_2} \quad:\iff\quad \gamma(\grad{\phi_1}) \subseteq \gamma(\grad{\phi_2})
\end{align*}

\newcommand{\dalpha}{\dot{\alpha}}

\subsection{Abstraction}
\begin{align*}
&\alpha(\overline{\phi}) &&= \min_{\sqsubseteq} {\{~ \grad{\phi} ~|~ \overline{\phi} \subseteq \gamma(\grad{\phi}) ~\}}\\
\end{align*}
Equivalent to:
\begin{align*}
&\alpha(\{ \phi \}) &&= \phi\\
&\alpha(\overline{\phi}) &&= \dalpha(\overline{\phi}) := \sup_{\sqsubseteq} {\{~ \withqm{\phi} ~|~ \phi \in \overline{\phi} ~\}}\\
\end{align*}

Proved:
\begin{itemize}
	\item partial function
	\item sound
	\item optimal
	\item $\alpha(\gamma(\grad{\phi})) = \grad{\phi}$
    \item does this make $\langle \gamma, \alpha \rangle$ a (partial) “galois insertion”?
\end{itemize}

\subsection{Lifting functions}
Gradual lifting $\grad{f} : \grad{\phi} \rightarrow \grad{\phi}$ of a function $f : \phi \rightarrow \phi$:
$$\grad{f}(\grad{\phi}) := \alpha(\overline{f}(\gamma(\grad{\phi})))$$ 

This formal definition has drawbacks:
\begin{itemize}
    \item Calculations on infinite set (not implementable)
    \item Determine supremum of infinite set (not even clear if it exists)
\end{itemize}

Turns out above definition can be rewritten in an equivalent, computable way.

\subsubsection{Dominator Theory}
TODO: first tackle singleton case etc.

Theorem:\\
% $\gamma(\phi) = \biguplus_{i = 1..n} \gamma(\hat{\phi}_i)$
For every $\phi$, there exists a finite set of “dominators” $\dom(\phi)$, such that 
$$\gamma(\withqm{\phi}) = \bigcup_{\hat{\phi} \in \dom(f(\phi))} \gamma(\withqm{\hat{\phi}})$$
~\\

Consequence: 
\begin{align*}
\withqm{\phi} 
&= \alpha(\gamma(\withqm{\phi})) \\
&= \dalpha(\gamma(\withqm{\phi})) \\
&= \dalpha(\bigcup_{\hat{\phi} \in \dom(\phi)} \gamma(\withqm{\hat{\phi}})) \\
&= \dalpha(\bigcup_{\hat{\phi} \in \dom(\phi)} \{ \hat{\phi} \}) \\
&= \dalpha(\dom(\phi)) \\
&= \sup_{\sqsubseteq} {\{~ \withqm{\phi'} ~|~ \phi' \in \dom(\phi) ~\}}
\end{align*}
~\\
Analogous, for monotonic $f$: 
\begin{align*}
&~~~~ \alpha(\overline{f}(\gamma(\withqm{\phi}))) \\
&= \dalpha(\overline{f}(\gamma(\withqm{\phi}))) \\
&= \dalpha(\overline{f}(\bigcup_{\hat{\phi} \in \dom(\phi)} \gamma(\withqm{\hat{\phi}}))) \\
&= \dalpha(\overline{f}(\bigcup_{\hat{\phi} \in \dom(\phi)} \{ \hat{\phi} \})) \\
&= \dalpha(\overline{f}(\dom(\phi))) \\
&= \sup_{\sqsubseteq} {\{~ \withqm{f(\phi')} ~|~ \phi' \in \dom(\phi) ~\}}
\end{align*}

~\\
Re-definition of gradual lifting:
$$\grad{f}(\phi) := f(\phi)$$ 
$$\grad{f}(\withqm{\phi}) := \alpha(\overline{f}(\gamma(\withqm{\phi}))) = \dalpha(\overline{f}(\dom(\phi)))$$ 

In terms of implementation: At least no more infinite sets, need to calculating supremum remains.\\

Interesting observation:
$$\grad{f}(\withqm{\hat{\phi}}) = \dalpha(\overline{f}(\dom(\hat{\phi}))) = \dalpha(\overline{f}(\{ \hat{\phi} \})) = \dalpha(\{ f(\hat{\phi}) \}) = \,\,\withqm{f(\hat{\phi})}$$

This observation raises the question whether it is possible to generalize the equality to work with arbitrary formulas, getting rid of $\dalpha$ and calculating a supremum entirely.

%Further observations:

%~\\
%Lemmas:
%\begin{itemize}
%    \item $\dom(\phi) \subseteq \gamma(\withqm{\phi})$
%    \item $\max \gamma(\withqm{\hat{\phi}}) = \hat{\phi}$
%    \item $f : \phi \rightarrow \phi \text{ monotonic }  \implies  \max \overline{\phi} = \phi'  \implies  \max \overline{f(\phi)} = f(\phi')$
%    \item $\max \overline{\phi} = \phi'  \implies  \alpha(\overline{\phi}) \in \{ \withqm{\phi'}, \phi' \}$
%\end{itemize}

%~\\
%Also:\\
%$\alpha(\overline{f}(\gamma(\phi))) =\, f(\phi)$

\subsubsection{Generalization: Auto-liftable functions}

Goal:
Get a definition of $\grad{f}$ that is even easier to handle and implement.
Therefore we want to investigate whether, or under which circumstances 
$$\grad{f}(\grad{\phi}) = f(\grad{\phi}) \quad\quad\text{(i.e. $f$ applied to the static part of $\grad{\phi}$)}$$
holds.

We call functions $f$ satisfying above equality “auto-liftable”.

% always holds for singletons

Counterexamples:
\begin{itemize}
    \item $f(\phi) = \acc(x.f) * \phi$\\
    $$\grad{f}(\withqm{x.f = 3}) = \,\,\withqm{\texttt{false}} \neq \,\,\withqm{\acc(x.f) * (x.f = 3)} = f(\withqm{(x.f = 3)})$$
    Cause: $\gamma(\withqm{x.f = 3})$ only contains self-framed formulas, so access to $x.f$ is always included. Adding it another time results in duplicate access and therefore unsatisfiable formulas.
    \item $f(\phi) = $ remove all terms containing $x$\\
    $$\grad{f}(\withqm{a = 3}) = \,\,? \neq \,\,\withqm{(a = 3)} = f(\withqm{(a = 3)})$$
    Cause: $(a = x) * (x = 3) \in \gamma(\withqm{a = 3})$ and $f((a = x) * (x = 3)) = \true$. Abstracting from a (non-singleton) set that contains $\true$ yields $?$.
    \end{itemize}

~\\
What is necessary to generalize this as $\alpha(\overline{f}(\gamma(\withqm{\phi}))) =\, \withqm{f(\phi)}$?

\begin{align*}
\forall \phi' \in \gamma(\withqm{f(\phi)}), \exists \phi'' \in \gamma(\withqm{\phi}), \phi' &\in \gamma(\withqm{f(\phi'')}) \\
\implies \\
\forall \phi' \in \gamma(\withqm{f(\phi)}), \exists \phi'' \in \gamma(\withqm{\phi}), \withqm{\phi'} &\sqsubseteq \,\,\withqm{f(\phi'')} \\
\implies \\
\forall \phi' \in \dom(f(\phi)), \exists \phi'' \in \dom(\phi), \withqm{\phi'} &\sqsubseteq \,\,\withqm{f(\phi'')} \\
\implies \\
\forall \phi' \in \dom(f(\phi)), \withqm{\phi'} &\sqsubseteq \sup_{\sqsubseteq} {\{~ \withqm{f(\phi')} ~|~ \phi' \in \dom(\phi) ~\}} \\
\iff \\
\sup_{\sqsubseteq} {\{~ \withqm{\phi'} ~|~ \phi' \in \dom(f(\phi)) ~\}} &\sqsubseteq \sup_{\sqsubseteq} {\{~ \withqm{f(\phi')} ~|~ \phi' \in \dom(\phi) ~\}} \\
\iff \\
\withqm{f(\phi)}  &\sqsubseteq \alpha(\overline{f}(\gamma(\withqm{\phi})))
\end{align*}

\begin{align*}
\withqm{f(\phi)} &\sqsubseteq \,\,\withqm{f(\phi)} \\
\implies \\
\forall \phi' \in \dom(\phi), \withqm{f(\phi')} &\sqsubseteq \,\,\withqm{f(\phi)} \\
\iff \\
\sup_{\sqsubseteq} {\{~ \withqm{f(\phi')} ~|~ \phi' \in \dom(\phi) ~\}} &\sqsubseteq \,\,\withqm{f(\phi)} \\
\iff \\
\alpha(\overline{f}(\gamma(\withqm{\phi}))) &\sqsubseteq \,\,\withqm{f(\phi)}
\end{align*}

For a function $f$ to be auto-liftable, the following properties are sufficient:
\begin{itemize}
    \item Monotonicity
    \item $\forall \phi' \in \gamma(\withqm{f(\phi)}), \exists \phi'' \in \gamma(\withqm{\phi}), \phi' \in \gamma(\withqm{f(\phi'')})$ 
\end{itemize}

\subsubsection{Liftable composition}
Given liftable functions $f$ and $g$, is $g \circ f$ liftable?
Monotonicity is obviously preserved.

Other condition:
\begin{align*}
\withqm{g(f(\phi))} \sqsubseteq \alpha(\overline{g}(\gamma(\withqm{f(\phi)}))) &\,\wedge
\,\,\withqm{f(\phi)} \sqsubseteq \alpha(\overline{f}(\gamma(\withqm{\phi}))) \\
\implies \\
\withqm{g(f(\phi))} \sqsubseteq \alpha(\overline{g}(\gamma(\withqm{f(\phi)}))) &\wedge
\alpha(\gamma(\withqm{f(\phi)})) \sqsubseteq \alpha(\overline{f}(\gamma(\withqm{\phi}))) \\
\implies \\
\withqm{g(f(\phi))} \sqsubseteq \alpha(\overline{g}(\gamma(\withqm{f(\phi)}))) &\wedge
\alpha(\overline{g}(\gamma(\withqm{f(\phi)}))) \sqsubseteq \alpha(\overline{g}(\overline{f}(\gamma(\withqm{\phi})))) \\
\implies \\
\withqm{g(f(\phi))} &\sqsubseteq \alpha(\overline{g}(\overline{f}(\gamma(\withqm{\phi})))) \\
\implies \\
\withqm{(g \circ f)(\phi)} &\sqsubseteq \alpha(\overline{(g \circ f)}(\gamma(\withqm{\phi}))) \\
\end{align*}


%\subsection{$(\phi, \implies)$ is semilattice}
%According to the definition of $\implies$ via $\subseteq$, we define
%\begin{align*}
%\phi_a \sqcap \phi_b = \phi_c   \quad :\iff \quad   \valphi{\phi_a} \cap \valphi{\phi_b} = \valphi{\phi_c}
%\end{align*}

%The question is, whether such $\phi_c$ always exists.



%\subsection{Concretization D (as in denotational)}
%\begin{align*}
%&\gamma(\phi) ~&&= \{~ \valphi{\phi} ~\} \\
%&\gamma(\withqm{\phi}) ~&&= \{~ \valphi{\phi\:*\:\phi_x} ~|~ \exists \phi_x : \valphi{\phi\:*\:\phi_x} \neq \emptyset ~\} \\
%& ~&&= \{~ \valphi{\phi'} ~|~ \exists \phi' : \emptyset \neq \valphi{\phi'} \wedge \phi' \implies \phi ~\} \\
%& ~&&= \{~ \valphi{\phi'} ~|~ \exists \phi' : \emptyset \neq \valphi{\phi'} \wedge \valphi{\phi'} \subseteq \valphi{\phi} ~\} \\
%\end{align*}

\subsection{Gradual Lifting}
\subsubsection{Self framing}
\begin{mathpar}
\inferrule* [Right=GSfrmNonGrad]
{A \sfrmphi \phi}
{A ~\grad{\sfrmphi}~ \phi}
\end{mathpar}

\begin{mathpar}
\inferrule* [Right=GSfrmGrad]
{~}
{A ~\grad{\sfrmphi}~ ?\:*\:\phi}
\end{mathpar}

\subsubsection{Implication}
\begin{mathpar}
\inferrule* [Right=GImplNonGrad]
{\phi_1 \implies \phi_2}
{\phi_1 ~\grad{\implies}~ \grad{\phi_2}}
\end{mathpar}

\begin{mathpar}
\inferrule* [Right=GImplGrad]
{\hat{\phi_m} \implies \phi_2 \\
 \hat{\phi_m} \implies \phi_1}
{?\:*\:\phi_1 ~\grad{\implies}~ \grad{\phi_2}}
\end{mathpar}

%Remark: Whether second argument is gradual or not seems to be irrelevant. Interestingly, all of our later uses will also pass non-gradual formulas as second argument. Maybe the natural lifting of this predicate should only lift on first argument in the first place?

$\hat{\phi_m}$ is evidence! \\


\textbf{Consistent transitivity}

While $\implies$ is transitive, $\grad{\implies}$ is generally not.

But maybe not even necessary with smarter hoare rules?

%\subsubsection{Free Variable}
%\begin{mathpar}
%\inferrule* [Right=GNotInFV]
%{x \not\in FV(\phi)}
%{x \not\in FV(\grad{\phi})}
%\end{mathpar}

\subsubsection{Equality}
\begin{mathpar}
\inferrule* [Right=GEqStatic]
{\phi_1 = \phi_2}
{\phi_1 \approx \phi_2}
\end{mathpar}

\begin{mathpar}
\inferrule* [Right=GEqGradual]
{
\text{at least one of $\grad{\phi_1}$ or $\grad{\phi_2}$ contains $?$}
\\\\
\grad{\phi_1} \grad{\implies} \grad{\phi_2} \\
\grad{\phi_2} \grad{\implies} \grad{\phi_1}
}
{\grad{\phi_1} \approx \grad{\phi_2}}
\end{mathpar}

\subsubsection{Append}
\begin{align*}
&\text{by definition:}\\
&\grad{\phi} ~\grad{*}~ \phi_p = \alpha(\gamma(\grad{\phi}) \overline{*} \phi_p)
&~\\\\
&\text{equivalent to:}\\
&\grad{\phi} ~\grad{*}~ \phi_p = \grad{\phi} * \phi_p
      && \text{if~} \forall \hat{\phi_1}, (\hat{\phi_1} \implies \phi * \phi_p) \implies 
                    \exists \hat{\phi_2}, (\hat{\phi_2} \implies \phi \wedge \hat{\phi_1} \implies \hat{\phi_2} * \phi_p) \\
&~
      && \text{if~} \forall \hat{\phi_1} \in \gamma(\grad{\phi} * \phi_p), 
                    \exists \hat{\phi_2} \in \gamma(\grad{\phi}), \hat{\phi_1} \implies \hat{\phi_2} * \phi_p \\
&\grad{\phi} ~\grad{*}~ \phi_p \textit{~undefined}
      && \text{otherwise}
\end{align*}
% (forall p'',(good p'' /\ phiImplies p'' (snd gp1 ++ p)) ->
% exists p' , good p'  /\ phiImplies p'  (snd gp1) /\ phiImplies p'' (p' ++ p))


\subsection{Gradual Hoare: minimal static rule approach}

\newcommand{\imp}[1]{\textcolor{red}{#1}}
Example:

\begin{mathpar}
\inferrule* [Right=GHVarAssign]
{
\imp{\epsilon \vdash\,} \grad{\phi} \grad{\implies} \grad{\phi'}
\\\imp{\emptyset \sfrmphi \grad{\phi'}}
\\\imp{x \not \in FV(\grad{\phi'})}
\\x \not \in FV(e)
\\\imp{\epsilon \vdash\,} \grad{\phi} \vdash x : T
\\\imp{\epsilon \vdash\,} \grad{\phi} \vdash e : T
\\\imp{\epsilon \vdash\,} \staticFP {\grad{\phi'}} \sfrme e
}
{\hoare {\grad{\phi}} {x := e} {\grad{\phi'} * (x = e)}}
\end{mathpar}

Collapsing (hidden) gradual implications into a single one:
\begin{mathpar}
\inferrule* [Right=GHVarAssign]
{
\imp{\epsilon \vdash\,} \grad{\phi} \grad{\implies} (x : T) * \hasTypeFormula e T C * \grad{\phi'}
\\\imp{\emptyset \sfrmphi \hasTypeFormula e T C * \grad{\phi'}}
\\\imp{x \not \in FV(\grad{\phi'})}
\\x \not \in FV(e)
\\\hasTypePremise e T C
}
{\hoare {\grad{\phi}} {x := e} {\hasTypeFormula e T C * \grad{\phi'} * (x = e)}}
\end{mathpar}

When shifting implication responsibility to GHSec:
\begin{mathpar}
\inferrule* [Right=GHVarAssign]
{
\imp{x \not \in FV(\grad{\phi'})}
\\x \not \in FV(e)
\\\hasTypePremise e T C
}
{\hoare {(x : T) * \hasTypeFormula e T C * \grad{\phi'}} {x := e} {\hasTypeFormula e T C * \grad{\phi'} * (x = e)}}
\end{mathpar}

Example derivation:
\begin{align*}
\\& \{(x : T) * \textcolor{blue}{(y : C) * \acc(y.a) * \acc(y.a.b) * \acc(y.a.b.c)} * \grad{\phi'}\}
\\& \{(x : T) * \textcolor{blue}{\hasTypeFormula {y.a.b.c} T C} * \grad{\phi'}\}
\\& x := y.a.b.c;	\quad\quad\quad\quad\textcolor{gray}{\parbox{14cm}{
\textcolor{red}{$x \not \in FV(\grad{\phi'})$}\\
$x \not \in FV(y.a.b.c)$\\
$\hasTypePremise {y.a.b.c} T C =
~\vdash C_y = C 	\,\,\wedge 
~\vdash C_y.a : C_a \,\,\wedge
~\vdash C_a.b : C_b \,\,\wedge
~\vdash C_b.c : T$}}
\\& \{\textcolor{blue}{\hasTypeFormula {y.a.b.c} T C} * \grad{\phi'} * (x = y.a.b.c)\}
\\& \{\textcolor{blue}{(y : C) * \acc(y.a) * \acc(y.a.b) * \acc(y.a.b.c)} * \grad{\phi'} * (x = y.a.b.c)\}
\end{align*}

\subsubsection{GHFieldAssign}

\begin{mathpar}
\inferrule* [Right=GHFieldAssign]
{
\sfrmphi \phi \\
\vdash C.f : T \\
\grad{\phi_1} \approx {(x : C) * (y : T) * (x \neq \vnull) * \phi * \acc(x.f)} \\
\grad{\phi_2} \approx {(x : C) * \acc(x.f) * (x \neq \vnull) * (x.f = y) * \phi}
}
{\ghoare
{\grad{\phi_1}} 
{x.f := y} 
{\grad{\phi_2}}}
\end{mathpar}

\subsubsection{GHSec - sound but obviously not complete!}

\begin{mathpar}
\inferrule* [Right=GHSec]
{
 \ghoare {\grad{\phi_p}} {s_1} {\grad{\phi_{q1}}} \\ 
 \phi_{q1} \implies \phi_{q2} \\
 \emptyset \sfrmphi \phi_{q2} \\
 \ghoare {\phi_{q2}} {s_2} {\grad{\phi_r}}
}
{\ghoare {\grad{\phi_p}} {s_1;s_2} {\grad{\phi_r}}}
\end{mathpar}

\subsection{Gradual Hoare: minimal HSec approach (implications per rule)}

\begin{mathpar}
\inferrule* [Right=HFieldAssign]
{
\sfrmphi \phi \\
\vdash C.f : T \\\\
\phi_1 \implies {(x : C) * (y : T) * \phi * \acc(x.f)} \\
\phi_2 = {(x : C) * \acc(x.f) * (x.f = y) * \phi}
}
{\hoare
{\phi_1}
{x.f := y}
{\phi_2}}
\end{mathpar}

\begin{mathpar}
\inferrule* [Right=GHFieldAssign]
{
\sfrmphi \phi \\
\vdash C.f : T \\\\
\grad{\phi_1} \grad{\implies} {(x : C) * (y : T) * \phi * \acc(x.f)} \\
\grad{\phi_2} \approx {(x : C) * \acc(x.f) * (x.f = y) * \phi}
}
{\ghoare
{\grad{\phi_1}}
{x.f := y}
{\grad{\phi_2}}}
\end{mathpar}

Note: With this alternative rule design $\grad{\implies}$ is consistently used with static formulas as second argument. 
This plays nicely with the fact that $\grad{\implies}$ does not care about the gradualness of that argument.
Might make sense to define lifting of $\implies$ as lifting on only the first parameter in the first place.\\

\textbf{Minimum runtime checks}: For $\grad{\phi_1} \grad{\implies} \grad{\phi_2}$ to hold at runtime, practically just $\phi_2$ needs to hold. So that would be a valid assertion to check. Yet, we know statically that $\phi_1$ holds, so we can remove everything from the runtime check that is implied by $\phi_1$.
So in a sense, we only need to check $\phi_2 \backslash \phi_1$ at runtime (the operator can be an approximation).


\subsection{Gradual Hoare: deterministic approach}
\subsubsection{HFieldAssign}

\begin{mathpar}
\inferrule* [Right=HFieldAssign]
{
\vdash C.f : T \\\\
\phi_1 \implies {(x : C) * (y : T) * \acc(x.f)} \\
\phi_2 = {(x : C) * \acc(x.f) * (x.f = y) * \phi_1[\textbf{w/o } \acc(x.f)]}
}
{\hoare
{\phi_1}
{x.f := y}
{\phi_2}}
\end{mathpar}

Note: $\phi[\textbf{w/o } \acc(x.f)]$ removes $\acc(x.f)$ and all uses of $x.f$ from $\phi$. The result is self-framed given that $\phi$ is.\\

\textbf{Attention}:
This version is weaker than the other (pairwise equivalent) versions of HFieldAssign!

Explanation: Above operator may remove more information than necessary from $\phi$.

Example:
\begin{itemize}
\item Given: $\phi_1 = \acc(x.f) * (x.f = a) * (x.f = b)$
\item Goal: $\phi_2 \implies (a = b)$
\item \textbf{not provable} with this deterministic version of HFieldAssign
\item \textbf{provable} with all other versions
\end{itemize}

Probably it's possible to apply the operator without information loss after expanding formula using equalities (transitive hull).

\subsubsection{GHFieldAssign}
(= gradual lifting of GHFieldAssign as function)

\begin{mathpar}
\inferrule* [Right=GHFieldAssign]
{
\grad{\phi_2} = \alpha(\{
\phi_2
~|~
\phi_1 \in \gamma(\grad{\phi_1})
~\wedge~
\hoare
{\phi_1}
{x.f := y}
{\phi_2}
~
\})
}
{
\ghoare
{\grad{\phi_1}}
{x.f := y}
{\grad{\phi_2}}
}
\end{mathpar}

Which should be equivalent to this:
\begin{mathpar}
\inferrule* [Right=GHFA1]
{
\vdash C.f : T \\\\
\phi_1 \implies {(x : C) * (y : T) * \acc(x.f)} \\\\
\phi_2 = {(x : C) * (y : T) * \acc(x.f) * (x.f = y) * \phi_1[\textbf{w/o } \acc(x.f)]}
}
{\ghoare
{\phi_1}
{x.f := y}
{\phi_2}}
\end{mathpar}

\begin{mathpar}
\inferrule* [Right=GHFA2]
{
\vdash C.f : T \\\\
\withqm{\phi_1} \grad{\implies}_{\phi_m} {(x : C) * \acc(x.f)} \\\\
\phi_2 = {(x : C) * \acc(x.f) * (x.f = y) * \phi_m[\textbf{w/o } \acc(x.f)]}
}
{\ghoare
{\withqm{\phi_1}}
{x.f := y}
{\withqm{\phi_2}}}
\end{mathpar}

Which should be summarizable as this:
\begin{mathpar}
\inferrule* [Right=GHFA]
{
\vdash C.f : T \\\\
\grad{\phi_1} \grad{\implies}_{\grad{\phi_m}} {(x : C) * (y : T) * \acc(x.f)} \\\\
\grad{\phi_2} = {(x : C) * \acc(x.f) * (x.f = y) * \grad{\phi_m}[\textbf{w/o } \acc(x.f)]}
}
{\ghoare
{\grad{\phi_1}}
{x.f := y}
{\grad{\phi_2}}}
\end{mathpar}

Which for well-formed programs is equivalent to:
\begin{mathpar}
\inferrule* [Right=GHFA]
{
\vdash C.f : T \\\\
\phi_1 \implies {(x : C) * (y : T)} \\
\grad{\phi_1} \grad{\implies} {\acc(x.f)} \\\\
\grad{\phi_2} = {(x : C) * (y : T) * \acc(x.f) * (x.f = y) * \grad{\phi_1}[\textbf{w/o } \acc(x.f)]}
}
{\ghoare
{\grad{\phi_1}}
{x.f := y}
{\grad{\phi_2}}}
\end{mathpar}

Observations:
\begin{itemize}
\item $\grad{\phi_m}$ is the interior (first argument) of the implication, effectively the meet of first and second argument.
\item for the gradual rules to work, the \textbf{w/o}-operator \textbf{must} be implemented with minimal information loss
\end{itemize}


\subsection{Theorems}
\subsubsection{Soundness of $\alpha$}
\begin{align*}
&\forall \overline{\phi} : \overline{\phi} \subseteq \gamma(\alpha(\overline{\phi}))
\end{align*}
\subsubsection{Optimality of $\alpha$}
\begin{align*}
&\forall \overline{\phi}, \grad{\phi} : \overline{\phi} \subseteq \gamma(\grad{\phi}) \implies  \gamma(\alpha(\overline{\phi}))\subseteq
\gamma(\grad{\phi}) 
\end{align*}

\section{Theorems}
\subsection{Invariant $invariant(H, \rho, A_d, \phi)$}

\subsubsection{Phi valid}
\begin{align*}
    \sfrmphi {\phi}
\end{align*}

\subsubsection{Phi holds}
\begin{align*}
    \evalphix {H} {\rho} {A_d} {\phi}
\end{align*}

\subsubsection{Types preserved}
\begin{align*}
    \forall e, T : \sType {\phi} {e} {T}& \\
    \implies \dType {H} {\rho} {e} {T}&
\end{align*}

\subsubsection{Heap consistent}
\begin{align*}
\forall o, C, \mu, f, T :&~ 
H(o) = (C, \mu) \\
\implies&~ 
\texttt{fieldType}(C,f) = T \\
\implies&~
\dType {H} {\rho} {\mu(f)} {T}
\end{align*}

\subsubsection{Heap not total}
\begin{align*}
\exists o_{min} :&\\
\forall o \ge o_{min} :&~ o \not \in \texttt{dom}(H) \\
\wedge&~ \forall f, (o, f) \not \in A
\end{align*}

\subsection{Soundness}
\subsubsection{Progress}
\begin{align*}
\forall~ ... &:~~ \hoare {\phi_1} {s'} {\phi_2} 
\\ &\implies invariant(H_1, \rho_1, A_1, \phi_1)
\\ &\implies \exists H_2, \rho_2, A_2 : (H_1, (\rho_1, A_1, s' ; \overline{s}) \cdot S)
							\rightarrow^* (H_2, (\rho_2, A_2, \overline{s}) \cdot S)
\end{align*}

\subsubsection{Preservation}
\begin{align*}
\forall~ ... &:~~ \hoare {\phi_1} {s'} {\phi_2} 
\\ &\implies invariant(H_1, \rho_1, A_1, \phi_1)
\\ &\implies (H_1, (\rho_1, A_1, s' ; \overline{s}) \cdot S)
  \rightarrow^* (H_2, (\rho_2, A_2, \overline{s}) \cdot S)
\\ &\implies invariant(H_2, \rho_2, A_2, \phi_2)
\end{align*}

\end{document}